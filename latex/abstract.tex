There are more than 7,000 languages on our planet today, but they are not evenly distributed. There are over 100 languages in Vanuatu, but only one in S\={a}moa. Why might this be? This paper explores this question for one particular region: Remote Oceania. Remote Oceania comprises the eastern Solomon Islands (Temotu), Vanuatu, New Caledonia, Fiji, Polynesia and Micronesia. The region features large differences in language richness, with some islands having 20 languages and many only one. This paper explores one hypothesis as to why this might be: more levels of political complexity reduces language diversification. In this study, we evaluate the strength of this claim by modelling political complexity as a predictor of language richness together with other relevant factors such as time-depth, size of island, rainfall etc. The results show that political complexity has a significant effect, but that it is not robust. Taking into account phylogenetic non-independence in particular reduces the effect, suggesting that there are relevant unaccounted for variables which are phylogenetically structures. The paper discusses further limitations of the study and possible expansions in the research area. 