% latex table generated in R 4.3.1 by xtable 1.8-4 package
% Wed Aug 23 15:12:36 2023
\begin{table}[ht]
\centering
\begin{tabular}{p{5cm}p{2cm}p{2cm}p{2cm}p{2cm}}
  \toprule
Island group (overnight-sailing) & Island group (shared language) & Political complexity (EA033) & glottocodes & references \\ 
  \midrule
Aotearoa & Aotearoa & 2.000 & maor1246, maor1246 & Sahlins MD (1958) Social Stratification in Polynesia (Univ Washington Press, Seattle, WA). Buck PH (1952) The Coming of the Maori (Whitcombe and Tombs: Wellington, New Zealand). Kirch PV (1984) The Evolution of the Polynesian Chiefdoms (Cambridge Univ Press, Cambridge, UK). Van Meijl T (1995) Maori Socio-Political Organization in Pre- and Proto-History: On the Evolution of Post-Colonial Constructs. Oceania 65(4):304-322. \\ 
  Banaba & Tungaru and Tuvalu & 2.000 & gilb1244 & Lambert B (1966) The Economic Activities of a Gilbertese Chief. Political Anthropology, ed Schwartz MJ, Turner VW, Tuden A (Transaction Publishers, New Brunswick, NJ), pp 155-172. Lambert B (1975) Makin and the Outside World. Pacific Atoll Populations, ed Carroll V (Univ Hawaii Press, Honolulu, HI), pp 212-285. Lambert B (1991) Kiribati. Encyclopaedia of World Cultures (Vol II: Oceania) (G.K. Hall and Co, New York, NY), pp 120-124. Macdonald B (1982) Cinderellas of the Empire: Towards a History of Kiribati and Tuvalu (ANU Press, Canberra, Australia). \\ 
  Bellona + Rennelle & Mo-ava-mo-iki & 2.000 & renn1242, renn1242 & Birket-Smith K (1969) An Ethnological Sketch of Rennell Island, a Polynesian Outlier in Melanesia (2nd Ed) (Bianco Lunos Bogtrykkeri, Copenhagen, Denmark); Monberg T (1991) Bellona Island Beliefs and Rituals (Univ Hawaii Press, Honolulu, HI). \\ 
  Chuuk & Chuuk & 2.000 & chuu1238 & Goodenough WH (1991) Truk. Encyclopaedia of World Cultures (Vol II: Oceania) (G.K. Hall and Co, New York, NY), pp 351-354; Goodenough WH (2002) Under Heaven's Brow: Pre-Christian Religious Tradition in Chuuk (American Philosopical Society, Philadelphia, PA); (1960) Taro cultivation in Truk. Taro Cultivation Practices and Beliefs: Part II. The Eastern Carolines and the Marshall Islands, ed Young JE (Office of the Staff Anthropologist, Guam, GU), pp 70-98. \\ 
  East Futuna & East Futuna & 2.000 & east2447 & Kirch PV (1994) The Wet and the Dry: Irrigation and Agricultural Intensification in Polynesia (Univ Chicago Press, Chicago, IL); \\ 
  Fiji & Kadavu & 3.000 & kada1285 & Kuhlken R (2002) Intensive Agricultural Landscapes of Oceania. Journal of Cultural Geography 19(2):161-195. 80) Scarr D (1984) Fiji: A Short History (George Allen and Unwin, Sydney, Australia). 81) Walter MAHB (1978) An examination of hierarchical notions in Fijian society: A test case for the applicability of the term ‘chief’. Oceania 49(1):1-19.; Hocart, A. M. 1929. Lau Islands, Fiji. (Bull. Bishop Mus., 62.) 1-240pp. Quain, Buell H. 1948. Fijian village. Chicago: University of Chicago Press. Thompson, L. 1940. Southern Lau, Fiji. (Bull. Bishop Mus., 162.) Thompson, Laura. 1940. Fijian frontier. (Studies of the Pacific.) San Francisco: Institute of Pacific Relations.; Kuhlken R (2002) Intensive Agricultural Landscapes of Oceania. Journal of Cultural Geography 19(2):161-195. 80) Scarr D (1984) Fiji: A Short History (George Allen and Unwin, Sydney, Australia). 81) Walter MAHB (1978) An examination of hierarchical notions in Fijian society: A test case for the applicability of the term ‘chief’. Oceania 49(1):1-19. \\ 
  Fiji & Lau & 3.000 & laua1243 & Kuhlken R (2002) Intensive Agricultural Landscapes of Oceania. Journal of Cultural Geography 19(2):161-195. 80) Scarr D (1984) Fiji: A Short History (George Allen and Unwin, Sydney, Australia). 81) Walter MAHB (1978) An examination of hierarchical notions in Fijian society: A test case for the applicability of the term ‘chief’. Oceania 49(1):1-19.; Hocart, A. M. 1929. Lau Islands, Fiji. (Bull. Bishop Mus., 62.) 1-240pp. Quain, Buell H. 1948. Fijian village. Chicago: University of Chicago Press. Thompson, L. 1940. Southern Lau, Fiji. (Bull. Bishop Mus., 162.) Thompson, Laura. 1940. Fijian frontier. (Studies of the Pacific.) San Francisco: Institute of Pacific Relations.; Kuhlken R (2002) Intensive Agricultural Landscapes of Oceania. Journal of Cultural Geography 19(2):161-195. 80) Scarr D (1984) Fiji: A Short History (George Allen and Unwin, Sydney, Australia). 81) Walter MAHB (1978) An examination of hierarchical notions in Fijian society: A test case for the applicability of the term ‘chief’. Oceania 49(1):1-19. \\ 
  Fiji & Viti Levu + Yasawa & 3.000 & sout2864, nort2843 & Kuhlken R (2002) Intensive Agricultural Landscapes of Oceania. Journal of Cultural Geography 19(2):161-195. 80) Scarr D (1984) Fiji: A Short History (George Allen and Unwin, Sydney, Australia). 81) Walter MAHB (1978) An examination of hierarchical notions in Fijian society: A test case for the applicability of the term ‘chief’. Oceania 49(1):1-19.; Hocart, A. M. 1929. Lau Islands, Fiji. (Bull. Bishop Mus., 62.) 1-240pp. Quain, Buell H. 1948. Fijian village. Chicago: University of Chicago Press. Thompson, L. 1940. Southern Lau, Fiji. (Bull. Bishop Mus., 162.) Thompson, Laura. 1940. Fijian frontier. (Studies of the Pacific.) San Francisco: Institute of Pacific Relations.; Kuhlken R (2002) Intensive Agricultural Landscapes of Oceania. Journal of Cultural Geography 19(2):161-195. 80) Scarr D (1984) Fiji: A Short History (George Allen and Unwin, Sydney, Australia). 81) Walter MAHB (1978) An examination of hierarchical notions in Fijian society: A test case for the applicability of the term ‘chief’. Oceania 49(1):1-19. \\ 
  Hawaii & Hawaii & 4.000 & hawa1245 & Kirch PV (1994) The Wet and the Dry: Irrigation and Agricultural Intensification in Polynesia (Univ Chicago Press, Chicago, IL). Kirch PV (2010) How Chiefs Became Kings: Divine Kingship and the Rise of Archaic States in Ancient Hawai'i (Univ California Press, Oakland, CA). \\ 
  Hereheretue & Tuamotu & 2.000 & tuam1242 & Emory KP (1975) Material Culture of the Tuamotu Archipelago (Bernice P Bishop Museum, Honolulu, HI). \\ 
  Kanaky & Kanaky (New Caledonia main island) & 1.000 & ajie1238 & Winslow, Don (1991) Ajie. Encyclopaedia of World Cultures (Vol II: Oceania) (G.K. Hall and Co, New York, NY), pp 9.; Young MW (1991) Goodenough Island. Encyclopaedia of World Cultures (Vol II: Oceania) (G.K. Hall and Co, New York, NY), pp 85-88.; Hadfield, E. 1920. Among the Natives of the Loyalty Group., Ray, S. 1917. The People and Language of Lifu, Loyalty Islands. Journ. Roy. Anth. Inst. 47. 239-322.; Dubois M (1984) Gens de Mar$\backslash$'\{e\} ($\backslash$'\{E\}ditions Anthropos, Paris, France). Guiart J (1952) L'Organisation Sociale et Politique Traditionelle $\backslash$`\{a\} Mar$\backslash$'\{e\}. (Institut Franc\{$\backslash$c c\}ais d'Oc$\backslash$'\{e\}anie, Noum$\backslash$'\{e\}a, New Caledonia). \\ 
  Kanaky & Kanaky (New Caledonia main island) & 3.000 & xara1244 & Winslow, Don (1991) Ajie. Encyclopaedia of World Cultures (Vol II: Oceania) (G.K. Hall and Co, New York, NY), pp 9.; Young MW (1991) Goodenough Island. Encyclopaedia of World Cultures (Vol II: Oceania) (G.K. Hall and Co, New York, NY), pp 85-88.; Hadfield, E. 1920. Among the Natives of the Loyalty Group., Ray, S. 1917. The People and Language of Lifu, Loyalty Islands. Journ. Roy. Anth. Inst. 47. 239-322.; Dubois M (1984) Gens de Mar$\backslash$'\{e\} ($\backslash$'\{E\}ditions Anthropos, Paris, France). Guiart J (1952) L'Organisation Sociale et Politique Traditionelle $\backslash$`\{a\} Mar$\backslash$'\{e\}. (Institut Franc\{$\backslash$c c\}ais d'Oc$\backslash$'\{e\}anie, Noum$\backslash$'\{e\}a, New Caledonia). \\ 
  Kanaky & Lifou & 2.000 & dehu1237 & Winslow, Don (1991) Ajie. Encyclopaedia of World Cultures (Vol II: Oceania) (G.K. Hall and Co, New York, NY), pp 9.; Young MW (1991) Goodenough Island. Encyclopaedia of World Cultures (Vol II: Oceania) (G.K. Hall and Co, New York, NY), pp 85-88.; Hadfield, E. 1920. Among the Natives of the Loyalty Group., Ray, S. 1917. The People and Language of Lifu, Loyalty Islands. Journ. Roy. Anth. Inst. 47. 239-322.; Dubois M (1984) Gens de Mar$\backslash$'\{e\} ($\backslash$'\{E\}ditions Anthropos, Paris, France). Guiart J (1952) L'Organisation Sociale et Politique Traditionelle $\backslash$`\{a\} Mar$\backslash$'\{e\}. (Institut Franc\{$\backslash$c c\}ais d'Oc$\backslash$'\{e\}anie, Noum$\backslash$'\{e\}a, New Caledonia). \\ 
  Kanaky & Nengone & 3.000 & neng1238 & Winslow, Don (1991) Ajie. Encyclopaedia of World Cultures (Vol II: Oceania) (G.K. Hall and Co, New York, NY), pp 9.; Young MW (1991) Goodenough Island. Encyclopaedia of World Cultures (Vol II: Oceania) (G.K. Hall and Co, New York, NY), pp 85-88.; Hadfield, E. 1920. Among the Natives of the Loyalty Group., Ray, S. 1917. The People and Language of Lifu, Loyalty Islands. Journ. Roy. Anth. Inst. 47. 239-322.; Dubois M (1984) Gens de Mar$\backslash$'\{e\} ($\backslash$'\{E\}ditions Anthropos, Paris, France). Guiart J (1952) L'Organisation Sociale et Politique Traditionelle $\backslash$`\{a\} Mar$\backslash$'\{e\}. (Institut Franc\{$\backslash$c c\}ais d'Oc$\backslash$'\{e\}anie, Noum$\backslash$'\{e\}a, New Caledonia). \\ 
  Kapingamarangi & Kapingamarangi & 1.000 & kapi1249 & Buck PH (1950) Material Culture of Kapingamarangi (Bernice P. Bishop Museum, Honolulu, HI). Emory KP (1965) Kapingamarangi: Social and Religious Life of a Polynesian Atoll (The Museum, Honolulu, HI). \\ 
  Kosrae & Kosrae & 3.000 & kosr1238 & Athens JS (2007) Prehistoric Population Growth on Kosrae, Eastern Caroline Islands. The Growth and Collapse of Pacific Island Societies, eds Kirch PV, Rallu J (Univ Hawaii Press, Honolulu, HI), pp 257-277. Graves MW (1986) Late Prehistoric Complexity on Lelū: Alternatives to Cordy’s Model. J Polyn Soc 95(4), 479-489. Peoples JG (1991) Kosrae. Encyclopaedia of World Cultures (Vol II: Oceania) (G.K. Hall and Co, New York, NY), pp 128-131. \\ 
  Laguas yan gåni & Laguas yan gåni & 1.000 & cham1312, cham1312, cham1312, caro1242, caro1242 & Cordy R (1983) Social stratification in the Mariana Islands. Oceania 53(3):272-276; Thompson L (1971) The Native Culture of the Marianas Islands (Bernice P Bishop Museum Bulletin, Honolulu, HI) (Originally published 1945).; Joseph, A., and V. F. Murray. 1951. Chamorros and Carolinians of Saipan: personality tests with an analysis of the Bender Gestalt test by Lauretta Bender. Cambridge: Harvard University Press. Spehr, A. 1954. Saipan. Fieldiana: Anth. 41. 1-383. \\ 
  Luangiua + Nukumanu & Luangiua & 1.000 & onto1237 & Sahlins MD (1958) Social Stratification in Polynesia (Univ Washington Press, Seattle, WA). Bayliss-Smith T (1974) Constraints on population growth: The case of the Polynesian Outlier atolls in the precontact period. Hum Ecol 2(4):259-295. Donner WW (1991) Ontong Java. Encyclopaedia of World Cultures (Vol II: Oceania) (G.K. Hall and Co, New York, NY), pp 253-255. \\ 
  Mangaia & Avaiki Nui (south) & 2.000 & raro1241 & Bellwood PS (1971) Varieties of Ecological Adaptation in the Southern Cook Islands. Archaeol Ocean 6(2):145-169. Buck PH (1934) Mangaian Society (Bernice P. Bishop Museum, Honolulu, HI). 240) Crocombe RG (1967) Ascendancy to dependency: the politics of Atiu. J Pac Hist 2(1):97-111. Gilson R, Crocombe R (1980) The Cook Islands 1820-1950 (Victoria Univ Press, Wellington, New Zealand). 242) Walter R (1996) Settlement pattern archaeology in the Southern Cook Islands: a review. J Polyn Soc 105(1):63-99. \\ 
  Mangareva & Mangareva & 2.000 & mang1401 & Buck PH (1971) Ethnology of Mangareva (Bernice P. Bishop Museum, Honolulu, HI) (Originally published 1938). Conte E, Kirch PV (2004) Archaeological Investigations in the Mangareva Islands (Gambier Archipelago), French Polynesia (Univ California, Berkeley, CA). Green RC and Weisler ML (2000) Mangarevan Archaeology: Interpretations using new data and 40 year old excavations to establish a sequence from 1200 to 1900 AD (Univ Otago, Dunedin, New Zealand). \\ 
  Marutea & Tuamotu & 2.000 & tuam1242 & Emory KP (1975) Material Culture of the Tuamotu Archipelago (Bernice P Bishop Museum, Honolulu, HI). \\ 
  Morane & Tuamotu & 2.000 & tuam1242 & Emory KP (1975) Material Culture of the Tuamotu Archipelago (Bernice P Bishop Museum, Honolulu, HI). \\ 
  Māngarongaro & Māngarongaro & 2.000 & penr1237 & Buck PH (1932) Ethnology of Tongareva (Bernice P. Bishop Museum, Honolulu, HI). Roscoe PB (1991) Tongareva. Encyclopaedia of World Cultures (Vol II: Oceania) (G.K. Hall and Co, New York, NY), pp 339-342. \\ 
  Napuka & Tuamotu & 2.000 & tuam1242 & Emory KP (1975) Material Culture of the Tuamotu Archipelago (Bernice P Bishop Museum, Honolulu, HI). \\ 
  Ngatik & Pohnpei & 3.000 & pohn1238 & Hanlon D (1988) Upon a Stone Altar: A History of the Island of Pohnpei to 1890 (Univ Hawaii Press, Honolulu HI). Haun AD (1984) Prehistoric Subsistence, Population, and Sociopolitical Evolution on Ponape, Micronesia. PhD thesis (Univ Oregon, Eugene, OR). Raynor WC, Fownes JH (1991) Indigenous agroforestry of Pohnpei. Agroforestry Systems 16:139-157. Riesenberg S (1968) The Native Polity of Ponape (Smithsonian Institution Press, Washington, DC). ; Hanlon, D. L. (2019). Upon a stone altar: A history of the island of Pohnpei to 1890. University of Hawaii Press. \\ 
  Ngā Pū Toru & Avaiki Nui (south) & 2.000 & raro1241 & Bellwood PS (1971) Varieties of Ecological Adaptation in the Southern Cook Islands. Archaeol Ocean 6(2):145-169. Buck PH (1934) Mangaian Society (Bernice P. Bishop Museum, Honolulu, HI). 240) Crocombe RG (1967) Ascendancy to dependency: the politics of Atiu. J Pac Hist 2(1):97-111. Gilson R, Crocombe R (1980) The Cook Islands 1820-1950 (Victoria Univ Press, Wellington, New Zealand). 242) Walter R (1996) Settlement pattern archaeology in the Southern Cook Islands: a review. J Polyn Soc 105(1):63-99. \\ 
  Niu + Tuvalu & Tungaru and Tuvalu & 2.000 & gilb1244, tuva1244 & Lambert B (1966) The Economic Activities of a Gilbertese Chief. Political Anthropology, ed Schwartz MJ, Turner VW, Tuden A (Transaction Publishers, New Brunswick, NJ), pp 155-172. Lambert B (1975) Makin and the Outside World. Pacific Atoll Populations, ed Carroll V (Univ Hawaii Press, Honolulu, HI), pp 212-285. Lambert B (1991) Kiribati. Encyclopaedia of World Cultures (Vol II: Oceania) (G.K. Hall and Co, New York, NY), pp 120-124. Macdonald B (1982) Cinderellas of the Empire: Towards a History of Kiribati and Tuvalu (ANU Press, Canberra, Australia).; Macdonald B (1982) Cinderellas of the Empire: Towards a History of Kiribati and Tuvalu (ANU Press, Canberra, Australia). Goldsmith M (1991) Tuvalu. Encyclopaedia of World Cultures (Vol II: Oceania) (G.K. Hall and Co, New York, NY), pp 354-357. \\ 
  Niue & Niue & 2.000 & niue1239 & Loeb EM (1978) History and Traditions of Niue (Bernice P Bishop Museum, Honolulu, HI). Smith SP (1983) Niue: The Island and Its People (The Polynesian Society, Suva, Fiji) (Originally published 1902-1903). Walter R, Anderson A (1995) Archaeology of Niue island: Initial Results. J Polyn Soc 104(4):471-481. \\ 
  Nukuoro & Nukuoro & 1.000 & nuku1260 & Carroll V (1966) Nukuoro Kinship. PhD thesis (Univ Chicago, Chicago, IL). Carroll V (1975) Demographic concepts and techniques for the study of small populations. Pacific Atoll Populations, ed Carrol V (Univ Hawaii Press, Honolulu, HI), pp 344-416. Eilers A (1934) Islands around Ponape : Kapingamarangi, Nukuoro, Ngatik, Mokil, Pingelap (Friederichsen, De Gruyter and Co, Hamburg, Germany). \\ 
  Nukutaveke 59 & Tuamotu & 2.000 & tuam1242 & Emory KP (1975) Material Culture of the Tuamotu Archipelago (Bernice P Bishop Museum, Honolulu, HI). \\ 
  Nukutaveke 61 & Tuamotu & 2.000 & tuam1242 & Emory KP (1975) Material Culture of the Tuamotu Archipelago (Bernice P Bishop Museum, Honolulu, HI). \\ 
  Palau & Palau & 2.000 & pala1344 & Force RW (1960) Leadership and Cultural Change in Palau (Chicago tural History Museum, Chicago, IL). \\ 
  Pohnpei & Pohnpei & 3.000 & pohn1238 & Hanlon D (1988) Upon a Stone Altar: A History of the Island of Pohnpei to 1890 (Univ Hawaii Press, Honolulu HI). Haun AD (1984) Prehistoric Subsistence, Population, and Sociopolitical Evolution on Ponape, Micronesia. PhD thesis (Univ Oregon, Eugene, OR). Raynor WC, Fownes JH (1991) Indigenous agroforestry of Pohnpei. Agroforestry Systems 16:139-157. Riesenberg S (1968) The Native Polity of Ponape (Smithsonian Institution Press, Washington, DC). ; Hanlon, D. L. (2019). Upon a stone altar: A history of the island of Pohnpei to 1890. University of Hawaii Press. \\ 
  Pukapuka & Pukapuka & 2.000 & puka1242 & Beaglehole, E., and P. Beaglehole. 1938. Ethnology of Pukapuka. Bull. Bishop Mus. 110. 1-419. Macgregor, G. 1935. Notes on the Ethnology of Pukapuka. Bishop Mus. Occas. Pap. 11. vi, 1-52. \\ 
  Raivavae & Raivavae & 2.000 & raiv1237 & Aitken RT (1971) Ethnology of Tubuai (Bernice P Bishop Museum, Honolulu, HI) (Originally published 1930). Bollt R (2008) Excavations in Peva Valley, Rurutu, Austral Islands (East Polynesia). Asian Perspect 47(1):158-187. Edwards E (2003) Archaeological Survey of Ra’ivavae (Bearsville Press, Los Osos, CA). \\ 
  Rakahanga-Manihiki & Rakahanga-Manihiki & 2.000 & raka1237 & Buck, P. H. 1932. Ethnology of Manihiki and Rakahanga. Bull. Bishop Mus. 99. 1-238. \\ 
  Rapa Nui & Rapa Nui & 2.000 & rapa1244 & Sahlins MD (1958) Social Stratification in Polynesia (Univ Washington Press, Seattle, WA). Kirch PV (1984) The Evolution of the Polynesian Chiefdoms (Cambridge Univ Press, Cambridge, UK). 209) M$\backslash$'\{e\}traux A (1971) Ethnology of Easter Island (Bernice P Bishop Museum, Honolulu, HI). \\ 
  Raro Matai + Nia Matai & Tahiti & 3.000 & tahi1242, tahi1242 & Oliver DL (1974) Ancient Tahitian Society (Volume 2: Social Relations) (Univ Hawaii Press: Honolulu, HI). Pages: 970-973 \\ 
  Rarotonga & Avaiki Nui (south) & 2.000 & raro1241 & Bellwood PS (1971) Varieties of Ecological Adaptation in the Southern Cook Islands. Archaeol Ocean 6(2):145-169. Buck PH (1934) Mangaian Society (Bernice P. Bishop Museum, Honolulu, HI). 240) Crocombe RG (1967) Ascendancy to dependency: the politics of Atiu. J Pac Hist 2(1):97-111. Gilson R, Crocombe R (1980) The Cook Islands 1820-1950 (Victoria Univ Press, Wellington, New Zealand). 242) Walter R (1996) Settlement pattern archaeology in the Southern Cook Islands: a review. J Polyn Soc 105(1):63-99. \\ 
  Ratak + Rālik & Ratak + Rālik & 3.000 & mars1254, mars1254 & Carucci LM (1991) Marshall Islands. Encyclopaedia of World Cultures (Vol II: Oceania) (G.K. Hall and Co, New York, NY), pp 191-194. Erdland A (1961) The Marshall Islanders: Life and Customs, Thought and Religion of a South Seas People (R. Neuse, Trans) (Human Relations Area Files, New Haven, CT) (Originally published 1914). Williamson I, Sabath MD (1982) Island Population, Land Area, and Climate: a Case Study of the Marshall Islands. Hum Ecol 10(1):71-84. \\ 
  Rotuma & Rotuma & 3.000 & rotu1241 & Gardiner JS (1898) The natives of Rotuma. The Journal of the Anthropological Institute of Great Britain and Ireland 27:396-435. Howard A (1963) Conservatism and non-traditional leadership in Rotuma. J Polyn Soc 72(2):65-77. Howard A (1991) Rotuma. Encyclopaedia of World Cultures (Vol II: Oceania) (G.K. Hall and Co, New York, NY), pp 280-283. \\ 
  Rurutu & Rurutu & 2.000 & ruru1237 & Aitken RT (1971) Ethnology of Tubuai (Bernice P Bishop Museum, Honolulu, HI) (Originally published 1930). Bollt R (2008) Excavations in Peva Valley, Rurutu, Austral Islands (East Polynesia). Asian Perspect 47(1):158-187. Edwards E (2003) Archaeological Survey of Ra’ivavae (Bearsville Press, Los Osos, CA). \\ 
  Rēkohou & Rēkohou & 2.000 & mori1267 & Sahlins MD (1958) Social Stratification in Polynesia (Univ Washington Press, Seattle, WA). Buck PH (1952) The Coming of the Maori (Whitcombe and Tombs: Wellington, New Zealand). Kirch PV (1984) The Evolution of the Polynesian Chiefdoms (Cambridge Univ Press, Cambridge, UK). Van Meijl T (1995) Maori Socio-Political Organization in Pre- and Proto-History: On the Evolution of Post-Colonial Constructs. Oceania 65(4):304-322. \\ 
  Sorol & Ulithi (greater) & 2.000 & ulit1238 & Lessa, W. A. 1950. The Ethnography of Ulithi Atoll. Unpublished Manuscript Ulithi (Micronesia). Lessa, William Armand. 1966. Ulithi: A Micronesian design for living. New York: Holt, Rinehart and Winston. \\ 
  Sāmoa & Sāmoa & 3.000 & samo1305 & Sahlins MD (1958) Social Stratification in Polynesia (Univ Washington Press, Seattle, WA). Buck PH (1930) Samoan Material Culture (Bernice P. Bishop Museum, Honolulu, HI). Keesing FM (1934) Modern Samoa: Its Government and Changing Life (Allen and Unwin Ltd, London, UK). 226) Watters RF (1958) Cultivation in Old Samoa. Economic Geography 43(4):338-351. \\ 
  Tatakoto + Reao & Tuamotu & 2.000 & tuam1242, tuam1242 & Emory KP (1975) Material Culture of the Tuamotu Archipelago (Bernice P Bishop Museum, Honolulu, HI). \\ 
  Te Henua ‘Enana + Te Fenua ’Enata & Te Henua ‘Enana & 1.000 & nort2845 & Sahlins MD (1958) Social Stratification in Polynesia (Univ Washington Press, Seattle, WA). \\ 
  Tokelau & Tokelau & 2.000 & toke1240 & Hooper A, Huntsman J (1973) A demographic history of the Tokelau Islands. J Polyn Soc 84(4):366-411. MacGregor G (1937) Ethnology of Tokelau Islands (Bernice P Bishop Museum, Honolulu, HI). \\ 
  Tonga & Tonga & 3.000 & tong1325 & Kirch PV (1984) The Evolution of the Polynesian Chiefdoms (Cambridge Univ Press, Cambridge, UK). Cummins HG (1977) Tongan Society at the Time of European Contact. Friendly Islands: A History of Tonga, ed Rutherford N (John Sands Ltd, Melbourne, Australia), pp 63-89. Ferdon EN (1987) Early Tonga (Univ Arizona Press, Tucson, AZ). \\ 
  Tuamotu + Puka-puka & Tuamotu & 2.000 & tuam1242, tuam1242 & Emory KP (1975) Material Culture of the Tuamotu Archipelago (Bernice P Bishop Museum, Honolulu, HI). \\ 
  Tungaru & Tungaru and Tuvalu & 2.000 & gilb1244 & Lambert B (1966) The Economic Activities of a Gilbertese Chief. Political Anthropology, ed Schwartz MJ, Turner VW, Tuden A (Transaction Publishers, New Brunswick, NJ), pp 155-172. Lambert B (1975) Makin and the Outside World. Pacific Atoll Populations, ed Carroll V (Univ Hawaii Press, Honolulu, HI), pp 212-285. Lambert B (1991) Kiribati. Encyclopaedia of World Cultures (Vol II: Oceania) (G.K. Hall and Co, New York, NY), pp 120-124. Macdonald B (1982) Cinderellas of the Empire: Towards a History of Kiribati and Tuvalu (ANU Press, Canberra, Australia). \\ 
  Tupuai & Tupuai & 2.000 & tubu1240 & Aitken RT (1971) Ethnology of Tubuai (Bernice P Bishop Museum, Honolulu, HI) (Originally published 1930). Bollt R (2008) Excavations in Peva Valley, Rurutu, Austral Islands (East Polynesia). Asian Perspect 47(1):158-187. Edwards E (2003) Archaeological Survey of Ra’ivavae (Bearsville Press, Los Osos, CA). \\ 
  Tureia & Tuamotu & 2.000 & tuam1242 & Emory KP (1975) Material Culture of the Tuamotu Archipelago (Bernice P Bishop Museum, Honolulu, HI). \\ 
  Tuvalu 31 & Tungaru and Tuvalu & 2.000 & tuva1244 & Macdonald B (1982) Cinderellas of the Empire: Towards a History of Kiribati and Tuvalu (ANU Press, Canberra, Australia). Goldsmith M (1991) Tuvalu. Encyclopaedia of World Cultures (Vol II: Oceania) (G.K. Hall and Co, New York, NY), pp 354-357. \\ 
  Tuvalu 32 & Tungaru and Tuvalu & 2.000 & tuva1244 & Macdonald B (1982) Cinderellas of the Empire: Towards a History of Kiribati and Tuvalu (ANU Press, Canberra, Australia). Goldsmith M (1991) Tuvalu. Encyclopaedia of World Cultures (Vol II: Oceania) (G.K. Hall and Co, New York, NY), pp 354-357. \\ 
  Ulithi + Yap & Ulithi (greater) & 2.000 & ulit1238 & Lessa, W. A. 1950. The Ethnography of Ulithi Atoll. Unpublished Manuscript Ulithi (Micronesia). Lessa, William Armand. 1966. Ulithi: A Micronesian design for living. New York: Holt, Rinehart and Winston.; Hunt, E. E., Jr., D. M. Schneider, N. R. Kidder, and W. D. Stevens. 1949. The Micronesians of Yap and Their Depopulation. Muller, W. 1917. Yap. (Ergebnisse der Südsee-Expedition 1908-1910, 2, B, iii.) In G. Thilenius (ed.) 1-380pp. Murdock, G. P., C. S. Ford, and J. W. M. Whiting. 1944. West Caroline Islands. 1-222pp. Salesius. 1906. Die Karolineninsel Jap. Schneider, David M. 1953. Yap Kinship Terminology and Kin Groups. American Anthropologist 55. 215-236.   Yapese   Schneider. 1957. Political Organization, Supernatural Sanctions and the Punishment for Incest on Yap. American Anthropologist 59. 791-800. Schneider, D. M. 1962. Double Descent on Yap. Journal of the Polynesian Society 71. 1-24. Tetens, A. 1958. Among the Savages of the South Seas. (Trans. F. M. Spoehr). Tetens, A., and J. Kubary. 1873. Die Carolineninsel Yap. Journal des Museum Godeffroy 1. 84-120. \\ 
  Ulithi + Yap & Yap & 3.000 & yape1248 & Lessa, W. A. 1950. The Ethnography of Ulithi Atoll. Unpublished Manuscript Ulithi (Micronesia). Lessa, William Armand. 1966. Ulithi: A Micronesian design for living. New York: Holt, Rinehart and Winston.; Hunt, E. E., Jr., D. M. Schneider, N. R. Kidder, and W. D. Stevens. 1949. The Micronesians of Yap and Their Depopulation. Muller, W. 1917. Yap. (Ergebnisse der Südsee-Expedition 1908-1910, 2, B, iii.) In G. Thilenius (ed.) 1-380pp. Murdock, G. P., C. S. Ford, and J. W. M. Whiting. 1944. West Caroline Islands. 1-222pp. Salesius. 1906. Die Karolineninsel Jap. Schneider, David M. 1953. Yap Kinship Terminology and Kin Groups. American Anthropologist 55. 215-236.   Yapese   Schneider. 1957. Political Organization, Supernatural Sanctions and the Punishment for Incest on Yap. American Anthropologist 59. 791-800. Schneider, D. M. 1962. Double Descent on Yap. Journal of the Polynesian Society 71. 1-24. Tetens, A. 1958. Among the Savages of the South Seas. (Trans. F. M. Spoehr). Tetens, A., and J. Kubary. 1873. Die Carolineninsel Yap. Journal des Museum Godeffroy 1. 84-120. \\ 
  Uvea (Wallis) & Uvea (Wallis) & 2.000 & wall1257 & Burrows EG (1971) Ethnology of Uvea (Wallis Island) (The Museum, Honolulu, HI) (Originally published 1937).  Pollock NJ (1995) The Power of Kava in Futuna and 'Uvea/Wallis. Canberra Anthropology 18(1-2):136-165. \\ 
  Vanuatu and Temotu & Ambae & 1.000 & east2443, west2513 & Bonnemaison, J. (1972). Système de grades et diff$\backslash$'\{e\}rences r$\backslash$'\{e\}gionales en Aoba (Nouvelles H$\backslash$'\{e\}brides). Cahiers ORSTOM. S$\backslash$'\{e\}rie Sciences Humaines, 9(1), 87-108.; Tonkinson R (1981) Church and Kastom in Southeast Ambrym. Vanuatu: Politics, Economics and Ritual in Island Melanesia, ed Allen M (Academic Press, Sydney, Australia), pp 237-267.; Humphreys CB (1926) Southern New Hebrides: An Ethnological Record (Cambridge Univ Press, Cambridge, UK); Spriggs M (1982) Taro Cropping Systems in the Southeast Asian-Pacific Region: Archaeological Evidence. Archaeol Ocean 17(1):7-15; Spriggs M (1986) Landscape, Land Use, and Political Transformation in Southern Melanesia. Island Societies: Archaeological Approaches to Evolution and Transformation, ed Kirch PV (Cambridge Univ Press, New York, NY), pp 6-19.; Feinberg R (1988) Socio-Spatial Symbolism and the Logic of Rank on Two Polynesian Outliers. Ethnology 27(3):291-310; Feinberg R (1991) Anuta. Encyclopaedia of World Cultures (Vol II: Oceania) (G.K. Hall and Co, New York, NY), pp 13-16.; Kirch PV (2002) Te Kai Paka-Anuta: Food in a Polynesian Outlier Society. Le Journal de la Soci$\backslash$'\{e\}t$\backslash$'\{e\} des Oc$\backslash$'\{e\}anistes 114-115:71-89.; Bonnemaison, J (1996) The Art of Power. In Bonnemaison (eds) Arts of Vanuatu. University of Hawaii Press.; Davenport WH (1969) Social organization notes on the Northern Santa Cruz Islands: the Main Reef Islands. Baessler-Archiv, Neue Folge 17(1):151-243.; Facey EE (1981) Hereditary chiefship in Nguna. Vanuatu: Politics, Economics and Ritual in Island Melanesia, ed Allen M (Academic Press, Sydney, Australia), pp 295-314. Facey EE (1991) Nguna. Encyclopaedia of World Cultures (Vol II: Oceania) (G.K. Hall and Co, New York, NY), pp 242-244.; Humphreys CB (1926) Southern New Hebrides: An Ethnological Record (Cambridge Univ Press, Cambridge, UK). Spriggs M, Wickler S (1989) Archaeological Research on Erromango: Recent Data on Southern Melanesian Prehistory. Bulletin of the Indo-Pacific Prehistory Association 9:68-91.; Capell A (1958) Culture and Language of Futuna and Aniwa, New Hebrides (Univ Sydney, Sydney, Australia).; Bonnemaison, J (1996) The Art of Power. In Bonnemaison (eds) Arts of Vanuatu. University of Hawaii Press.; Bonnemaison, J (1996) The Art of Power. In Bonnemaison (eds) Arts of Vanuatu. University of Hawaii Press.; Bonnemaison, J (1996) The Art of Power. In Bonnemaison (eds) Arts of Vanuatu. University of Hawaii Press.; Bonnemaison, J (1996) The Art of Power. In Bonnemaison (eds) Arts of Vanuatu. University of Hawaii Press.; Deacon, A. B. 1934. Malekula.; Bonnemaison, J (1996) The Art of Power. In Bonnemaison (eds) Arts of Vanuatu. University of Hawaii Press.; Bonnemaison, J (1996) The Art of Power. In Bonnemaison (eds) Arts of Vanuatu. University of Hawaii Press.; Bonnemaison, J (1996) The Art of Power. In Bonnemaison (eds) Arts of Vanuatu. University of Hawaii Press.; Lane, R. B. 1956. The Heathen Communities of Southeast Pentecost. Journal de la Soci‚te des Oceanistes 12. 139-180., Lane, R. B. 1965. The Melanesians of South Pentecost. In P. Lawrence and M. G. Meggitt (eds.), Gods, Ghosts and Men in Melanesia, 250-279. Lane, R. B., and B. S. Lane. 1957. Unpublished field notes.; Bonnemaison, J (1996) The Art of Power. In Bonnemaison (eds) Arts of Vanuatu. University of Hawaii Press.; Lindström, Lamont (1991) Ajie. Encyclopaedia of World Cultures (Vol II: Oceania) (G.K. Hall and Co, New York, NY), pp 314.; Kirch PV (1994) The Wet and the Dry: Irrigation and Agricultural Intensification in Polynesia (Univ Chicago Press, Chicago, IL). Sahlins MD (1958) Social Stratification in Polynesia (Univ Washington Press, Seattle, WA). Firth R (1939) Primitive Polynesian Economy (George Routledge and Sons, London, UK). Firth R (1959) Social Change in Tikopia: Re-Study of a Polynesian Community after a Generation (Allen and Unwin, London, UK). Firth R (1991) Tikopia. Encyclopaedia of World Cultures (Vol II: Oceania) (G.K. Hall and Co, New York, NY), pp 324-327.; Bonnemaison, J (1996) The Art of Power. In Bonnemaison (eds) Arts of Vanuatu. University of Hawaii Press.; Bonnemaison, J (1996) The Art of Power. In Bonnemaison (eds) Arts of Vanuatu. University of Hawaii Press. \\ 
  Vanuatu and Temotu & Ambrym & 1.000 & sout2859 & Bonnemaison, J. (1972). Système de grades et diff$\backslash$'\{e\}rences r$\backslash$'\{e\}gionales en Aoba (Nouvelles H$\backslash$'\{e\}brides). Cahiers ORSTOM. S$\backslash$'\{e\}rie Sciences Humaines, 9(1), 87-108.; Tonkinson R (1981) Church and Kastom in Southeast Ambrym. Vanuatu: Politics, Economics and Ritual in Island Melanesia, ed Allen M (Academic Press, Sydney, Australia), pp 237-267.; Humphreys CB (1926) Southern New Hebrides: An Ethnological Record (Cambridge Univ Press, Cambridge, UK); Spriggs M (1982) Taro Cropping Systems in the Southeast Asian-Pacific Region: Archaeological Evidence. Archaeol Ocean 17(1):7-15; Spriggs M (1986) Landscape, Land Use, and Political Transformation in Southern Melanesia. Island Societies: Archaeological Approaches to Evolution and Transformation, ed Kirch PV (Cambridge Univ Press, New York, NY), pp 6-19.; Feinberg R (1988) Socio-Spatial Symbolism and the Logic of Rank on Two Polynesian Outliers. Ethnology 27(3):291-310; Feinberg R (1991) Anuta. Encyclopaedia of World Cultures (Vol II: Oceania) (G.K. Hall and Co, New York, NY), pp 13-16.; Kirch PV (2002) Te Kai Paka-Anuta: Food in a Polynesian Outlier Society. Le Journal de la Soci$\backslash$'\{e\}t$\backslash$'\{e\} des Oc$\backslash$'\{e\}anistes 114-115:71-89.; Bonnemaison, J (1996) The Art of Power. In Bonnemaison (eds) Arts of Vanuatu. University of Hawaii Press.; Davenport WH (1969) Social organization notes on the Northern Santa Cruz Islands: the Main Reef Islands. Baessler-Archiv, Neue Folge 17(1):151-243.; Facey EE (1981) Hereditary chiefship in Nguna. Vanuatu: Politics, Economics and Ritual in Island Melanesia, ed Allen M (Academic Press, Sydney, Australia), pp 295-314. Facey EE (1991) Nguna. Encyclopaedia of World Cultures (Vol II: Oceania) (G.K. Hall and Co, New York, NY), pp 242-244.; Humphreys CB (1926) Southern New Hebrides: An Ethnological Record (Cambridge Univ Press, Cambridge, UK). Spriggs M, Wickler S (1989) Archaeological Research on Erromango: Recent Data on Southern Melanesian Prehistory. Bulletin of the Indo-Pacific Prehistory Association 9:68-91.; Capell A (1958) Culture and Language of Futuna and Aniwa, New Hebrides (Univ Sydney, Sydney, Australia).; Bonnemaison, J (1996) The Art of Power. In Bonnemaison (eds) Arts of Vanuatu. University of Hawaii Press.; Bonnemaison, J (1996) The Art of Power. In Bonnemaison (eds) Arts of Vanuatu. University of Hawaii Press.; Bonnemaison, J (1996) The Art of Power. In Bonnemaison (eds) Arts of Vanuatu. University of Hawaii Press.; Bonnemaison, J (1996) The Art of Power. In Bonnemaison (eds) Arts of Vanuatu. University of Hawaii Press.; Deacon, A. B. 1934. Malekula.; Bonnemaison, J (1996) The Art of Power. In Bonnemaison (eds) Arts of Vanuatu. University of Hawaii Press.; Bonnemaison, J (1996) The Art of Power. In Bonnemaison (eds) Arts of Vanuatu. University of Hawaii Press.; Bonnemaison, J (1996) The Art of Power. In Bonnemaison (eds) Arts of Vanuatu. University of Hawaii Press.; Lane, R. B. 1956. The Heathen Communities of Southeast Pentecost. Journal de la Soci‚te des Oceanistes 12. 139-180., Lane, R. B. 1965. The Melanesians of South Pentecost. In P. Lawrence and M. G. Meggitt (eds.), Gods, Ghosts and Men in Melanesia, 250-279. Lane, R. B., and B. S. Lane. 1957. Unpublished field notes.; Bonnemaison, J (1996) The Art of Power. In Bonnemaison (eds) Arts of Vanuatu. University of Hawaii Press.; Lindström, Lamont (1991) Ajie. Encyclopaedia of World Cultures (Vol II: Oceania) (G.K. Hall and Co, New York, NY), pp 314.; Kirch PV (1994) The Wet and the Dry: Irrigation and Agricultural Intensification in Polynesia (Univ Chicago Press, Chicago, IL). Sahlins MD (1958) Social Stratification in Polynesia (Univ Washington Press, Seattle, WA). Firth R (1939) Primitive Polynesian Economy (George Routledge and Sons, London, UK). Firth R (1959) Social Change in Tikopia: Re-Study of a Polynesian Community after a Generation (Allen and Unwin, London, UK). Firth R (1991) Tikopia. Encyclopaedia of World Cultures (Vol II: Oceania) (G.K. Hall and Co, New York, NY), pp 324-327.; Bonnemaison, J (1996) The Art of Power. In Bonnemaison (eds) Arts of Vanuatu. University of Hawaii Press.; Bonnemaison, J (1996) The Art of Power. In Bonnemaison (eds) Arts of Vanuatu. University of Hawaii Press. \\ 
  Vanuatu and Temotu & Aneityum & 2.000 & anei1239 & Bonnemaison, J. (1972). Système de grades et diff$\backslash$'\{e\}rences r$\backslash$'\{e\}gionales en Aoba (Nouvelles H$\backslash$'\{e\}brides). Cahiers ORSTOM. S$\backslash$'\{e\}rie Sciences Humaines, 9(1), 87-108.; Tonkinson R (1981) Church and Kastom in Southeast Ambrym. Vanuatu: Politics, Economics and Ritual in Island Melanesia, ed Allen M (Academic Press, Sydney, Australia), pp 237-267.; Humphreys CB (1926) Southern New Hebrides: An Ethnological Record (Cambridge Univ Press, Cambridge, UK); Spriggs M (1982) Taro Cropping Systems in the Southeast Asian-Pacific Region: Archaeological Evidence. Archaeol Ocean 17(1):7-15; Spriggs M (1986) Landscape, Land Use, and Political Transformation in Southern Melanesia. Island Societies: Archaeological Approaches to Evolution and Transformation, ed Kirch PV (Cambridge Univ Press, New York, NY), pp 6-19.; Feinberg R (1988) Socio-Spatial Symbolism and the Logic of Rank on Two Polynesian Outliers. Ethnology 27(3):291-310; Feinberg R (1991) Anuta. Encyclopaedia of World Cultures (Vol II: Oceania) (G.K. Hall and Co, New York, NY), pp 13-16.; Kirch PV (2002) Te Kai Paka-Anuta: Food in a Polynesian Outlier Society. Le Journal de la Soci$\backslash$'\{e\}t$\backslash$'\{e\} des Oc$\backslash$'\{e\}anistes 114-115:71-89.; Bonnemaison, J (1996) The Art of Power. In Bonnemaison (eds) Arts of Vanuatu. University of Hawaii Press.; Davenport WH (1969) Social organization notes on the Northern Santa Cruz Islands: the Main Reef Islands. Baessler-Archiv, Neue Folge 17(1):151-243.; Facey EE (1981) Hereditary chiefship in Nguna. Vanuatu: Politics, Economics and Ritual in Island Melanesia, ed Allen M (Academic Press, Sydney, Australia), pp 295-314. Facey EE (1991) Nguna. Encyclopaedia of World Cultures (Vol II: Oceania) (G.K. Hall and Co, New York, NY), pp 242-244.; Humphreys CB (1926) Southern New Hebrides: An Ethnological Record (Cambridge Univ Press, Cambridge, UK). Spriggs M, Wickler S (1989) Archaeological Research on Erromango: Recent Data on Southern Melanesian Prehistory. Bulletin of the Indo-Pacific Prehistory Association 9:68-91.; Capell A (1958) Culture and Language of Futuna and Aniwa, New Hebrides (Univ Sydney, Sydney, Australia).; Bonnemaison, J (1996) The Art of Power. In Bonnemaison (eds) Arts of Vanuatu. University of Hawaii Press.; Bonnemaison, J (1996) The Art of Power. In Bonnemaison (eds) Arts of Vanuatu. University of Hawaii Press.; Bonnemaison, J (1996) The Art of Power. In Bonnemaison (eds) Arts of Vanuatu. University of Hawaii Press.; Bonnemaison, J (1996) The Art of Power. In Bonnemaison (eds) Arts of Vanuatu. University of Hawaii Press.; Deacon, A. B. 1934. Malekula.; Bonnemaison, J (1996) The Art of Power. In Bonnemaison (eds) Arts of Vanuatu. University of Hawaii Press.; Bonnemaison, J (1996) The Art of Power. In Bonnemaison (eds) Arts of Vanuatu. University of Hawaii Press.; Bonnemaison, J (1996) The Art of Power. In Bonnemaison (eds) Arts of Vanuatu. University of Hawaii Press.; Lane, R. B. 1956. The Heathen Communities of Southeast Pentecost. Journal de la Soci‚te des Oceanistes 12. 139-180., Lane, R. B. 1965. The Melanesians of South Pentecost. In P. Lawrence and M. G. Meggitt (eds.), Gods, Ghosts and Men in Melanesia, 250-279. Lane, R. B., and B. S. Lane. 1957. Unpublished field notes.; Bonnemaison, J (1996) The Art of Power. In Bonnemaison (eds) Arts of Vanuatu. University of Hawaii Press.; Lindström, Lamont (1991) Ajie. Encyclopaedia of World Cultures (Vol II: Oceania) (G.K. Hall and Co, New York, NY), pp 314.; Kirch PV (1994) The Wet and the Dry: Irrigation and Agricultural Intensification in Polynesia (Univ Chicago Press, Chicago, IL). Sahlins MD (1958) Social Stratification in Polynesia (Univ Washington Press, Seattle, WA). Firth R (1939) Primitive Polynesian Economy (George Routledge and Sons, London, UK). Firth R (1959) Social Change in Tikopia: Re-Study of a Polynesian Community after a Generation (Allen and Unwin, London, UK). Firth R (1991) Tikopia. Encyclopaedia of World Cultures (Vol II: Oceania) (G.K. Hall and Co, New York, NY), pp 324-327.; Bonnemaison, J (1996) The Art of Power. In Bonnemaison (eds) Arts of Vanuatu. University of Hawaii Press.; Bonnemaison, J (1996) The Art of Power. In Bonnemaison (eds) Arts of Vanuatu. University of Hawaii Press. \\ 
  Vanuatu and Temotu & Anuta & 1.000 & anut1237 & Bonnemaison, J. (1972). Système de grades et diff$\backslash$'\{e\}rences r$\backslash$'\{e\}gionales en Aoba (Nouvelles H$\backslash$'\{e\}brides). Cahiers ORSTOM. S$\backslash$'\{e\}rie Sciences Humaines, 9(1), 87-108.; Tonkinson R (1981) Church and Kastom in Southeast Ambrym. Vanuatu: Politics, Economics and Ritual in Island Melanesia, ed Allen M (Academic Press, Sydney, Australia), pp 237-267.; Humphreys CB (1926) Southern New Hebrides: An Ethnological Record (Cambridge Univ Press, Cambridge, UK); Spriggs M (1982) Taro Cropping Systems in the Southeast Asian-Pacific Region: Archaeological Evidence. Archaeol Ocean 17(1):7-15; Spriggs M (1986) Landscape, Land Use, and Political Transformation in Southern Melanesia. Island Societies: Archaeological Approaches to Evolution and Transformation, ed Kirch PV (Cambridge Univ Press, New York, NY), pp 6-19.; Feinberg R (1988) Socio-Spatial Symbolism and the Logic of Rank on Two Polynesian Outliers. Ethnology 27(3):291-310; Feinberg R (1991) Anuta. Encyclopaedia of World Cultures (Vol II: Oceania) (G.K. Hall and Co, New York, NY), pp 13-16.; Kirch PV (2002) Te Kai Paka-Anuta: Food in a Polynesian Outlier Society. Le Journal de la Soci$\backslash$'\{e\}t$\backslash$'\{e\} des Oc$\backslash$'\{e\}anistes 114-115:71-89.; Bonnemaison, J (1996) The Art of Power. In Bonnemaison (eds) Arts of Vanuatu. University of Hawaii Press.; Davenport WH (1969) Social organization notes on the Northern Santa Cruz Islands: the Main Reef Islands. Baessler-Archiv, Neue Folge 17(1):151-243.; Facey EE (1981) Hereditary chiefship in Nguna. Vanuatu: Politics, Economics and Ritual in Island Melanesia, ed Allen M (Academic Press, Sydney, Australia), pp 295-314. Facey EE (1991) Nguna. Encyclopaedia of World Cultures (Vol II: Oceania) (G.K. Hall and Co, New York, NY), pp 242-244.; Humphreys CB (1926) Southern New Hebrides: An Ethnological Record (Cambridge Univ Press, Cambridge, UK). Spriggs M, Wickler S (1989) Archaeological Research on Erromango: Recent Data on Southern Melanesian Prehistory. Bulletin of the Indo-Pacific Prehistory Association 9:68-91.; Capell A (1958) Culture and Language of Futuna and Aniwa, New Hebrides (Univ Sydney, Sydney, Australia).; Bonnemaison, J (1996) The Art of Power. In Bonnemaison (eds) Arts of Vanuatu. University of Hawaii Press.; Bonnemaison, J (1996) The Art of Power. In Bonnemaison (eds) Arts of Vanuatu. University of Hawaii Press.; Bonnemaison, J (1996) The Art of Power. In Bonnemaison (eds) Arts of Vanuatu. University of Hawaii Press.; Bonnemaison, J (1996) The Art of Power. In Bonnemaison (eds) Arts of Vanuatu. University of Hawaii Press.; Deacon, A. B. 1934. Malekula.; Bonnemaison, J (1996) The Art of Power. In Bonnemaison (eds) Arts of Vanuatu. University of Hawaii Press.; Bonnemaison, J (1996) The Art of Power. In Bonnemaison (eds) Arts of Vanuatu. University of Hawaii Press.; Bonnemaison, J (1996) The Art of Power. In Bonnemaison (eds) Arts of Vanuatu. University of Hawaii Press.; Lane, R. B. 1956. The Heathen Communities of Southeast Pentecost. Journal de la Soci‚te des Oceanistes 12. 139-180., Lane, R. B. 1965. The Melanesians of South Pentecost. In P. Lawrence and M. G. Meggitt (eds.), Gods, Ghosts and Men in Melanesia, 250-279. Lane, R. B., and B. S. Lane. 1957. Unpublished field notes.; Bonnemaison, J (1996) The Art of Power. In Bonnemaison (eds) Arts of Vanuatu. University of Hawaii Press.; Lindström, Lamont (1991) Ajie. Encyclopaedia of World Cultures (Vol II: Oceania) (G.K. Hall and Co, New York, NY), pp 314.; Kirch PV (1994) The Wet and the Dry: Irrigation and Agricultural Intensification in Polynesia (Univ Chicago Press, Chicago, IL). Sahlins MD (1958) Social Stratification in Polynesia (Univ Washington Press, Seattle, WA). Firth R (1939) Primitive Polynesian Economy (George Routledge and Sons, London, UK). Firth R (1959) Social Change in Tikopia: Re-Study of a Polynesian Community after a Generation (Allen and Unwin, London, UK). Firth R (1991) Tikopia. Encyclopaedia of World Cultures (Vol II: Oceania) (G.K. Hall and Co, New York, NY), pp 324-327.; Bonnemaison, J (1996) The Art of Power. In Bonnemaison (eds) Arts of Vanuatu. University of Hawaii Press.; Bonnemaison, J (1996) The Art of Power. In Bonnemaison (eds) Arts of Vanuatu. University of Hawaii Press. \\ 
  Vanuatu and Temotu & Aore & 1.000 & aore1237 & Bonnemaison, J. (1972). Système de grades et diff$\backslash$'\{e\}rences r$\backslash$'\{e\}gionales en Aoba (Nouvelles H$\backslash$'\{e\}brides). Cahiers ORSTOM. S$\backslash$'\{e\}rie Sciences Humaines, 9(1), 87-108.; Tonkinson R (1981) Church and Kastom in Southeast Ambrym. Vanuatu: Politics, Economics and Ritual in Island Melanesia, ed Allen M (Academic Press, Sydney, Australia), pp 237-267.; Humphreys CB (1926) Southern New Hebrides: An Ethnological Record (Cambridge Univ Press, Cambridge, UK); Spriggs M (1982) Taro Cropping Systems in the Southeast Asian-Pacific Region: Archaeological Evidence. Archaeol Ocean 17(1):7-15; Spriggs M (1986) Landscape, Land Use, and Political Transformation in Southern Melanesia. Island Societies: Archaeological Approaches to Evolution and Transformation, ed Kirch PV (Cambridge Univ Press, New York, NY), pp 6-19.; Feinberg R (1988) Socio-Spatial Symbolism and the Logic of Rank on Two Polynesian Outliers. Ethnology 27(3):291-310; Feinberg R (1991) Anuta. Encyclopaedia of World Cultures (Vol II: Oceania) (G.K. Hall and Co, New York, NY), pp 13-16.; Kirch PV (2002) Te Kai Paka-Anuta: Food in a Polynesian Outlier Society. Le Journal de la Soci$\backslash$'\{e\}t$\backslash$'\{e\} des Oc$\backslash$'\{e\}anistes 114-115:71-89.; Bonnemaison, J (1996) The Art of Power. In Bonnemaison (eds) Arts of Vanuatu. University of Hawaii Press.; Davenport WH (1969) Social organization notes on the Northern Santa Cruz Islands: the Main Reef Islands. Baessler-Archiv, Neue Folge 17(1):151-243.; Facey EE (1981) Hereditary chiefship in Nguna. Vanuatu: Politics, Economics and Ritual in Island Melanesia, ed Allen M (Academic Press, Sydney, Australia), pp 295-314. Facey EE (1991) Nguna. Encyclopaedia of World Cultures (Vol II: Oceania) (G.K. Hall and Co, New York, NY), pp 242-244.; Humphreys CB (1926) Southern New Hebrides: An Ethnological Record (Cambridge Univ Press, Cambridge, UK). Spriggs M, Wickler S (1989) Archaeological Research on Erromango: Recent Data on Southern Melanesian Prehistory. Bulletin of the Indo-Pacific Prehistory Association 9:68-91.; Capell A (1958) Culture and Language of Futuna and Aniwa, New Hebrides (Univ Sydney, Sydney, Australia).; Bonnemaison, J (1996) The Art of Power. In Bonnemaison (eds) Arts of Vanuatu. University of Hawaii Press.; Bonnemaison, J (1996) The Art of Power. In Bonnemaison (eds) Arts of Vanuatu. University of Hawaii Press.; Bonnemaison, J (1996) The Art of Power. In Bonnemaison (eds) Arts of Vanuatu. University of Hawaii Press.; Bonnemaison, J (1996) The Art of Power. In Bonnemaison (eds) Arts of Vanuatu. University of Hawaii Press.; Deacon, A. B. 1934. Malekula.; Bonnemaison, J (1996) The Art of Power. In Bonnemaison (eds) Arts of Vanuatu. University of Hawaii Press.; Bonnemaison, J (1996) The Art of Power. In Bonnemaison (eds) Arts of Vanuatu. University of Hawaii Press.; Bonnemaison, J (1996) The Art of Power. In Bonnemaison (eds) Arts of Vanuatu. University of Hawaii Press.; Lane, R. B. 1956. The Heathen Communities of Southeast Pentecost. Journal de la Soci‚te des Oceanistes 12. 139-180., Lane, R. B. 1965. The Melanesians of South Pentecost. In P. Lawrence and M. G. Meggitt (eds.), Gods, Ghosts and Men in Melanesia, 250-279. Lane, R. B., and B. S. Lane. 1957. Unpublished field notes.; Bonnemaison, J (1996) The Art of Power. In Bonnemaison (eds) Arts of Vanuatu. University of Hawaii Press.; Lindström, Lamont (1991) Ajie. Encyclopaedia of World Cultures (Vol II: Oceania) (G.K. Hall and Co, New York, NY), pp 314.; Kirch PV (1994) The Wet and the Dry: Irrigation and Agricultural Intensification in Polynesia (Univ Chicago Press, Chicago, IL). Sahlins MD (1958) Social Stratification in Polynesia (Univ Washington Press, Seattle, WA). Firth R (1939) Primitive Polynesian Economy (George Routledge and Sons, London, UK). Firth R (1959) Social Change in Tikopia: Re-Study of a Polynesian Community after a Generation (Allen and Unwin, London, UK). Firth R (1991) Tikopia. Encyclopaedia of World Cultures (Vol II: Oceania) (G.K. Hall and Co, New York, NY), pp 324-327.; Bonnemaison, J (1996) The Art of Power. In Bonnemaison (eds) Arts of Vanuatu. University of Hawaii Press.; Bonnemaison, J (1996) The Art of Power. In Bonnemaison (eds) Arts of Vanuatu. University of Hawaii Press. \\ 
  Vanuatu and Temotu & Duff and Reef Islands & 1.000 & ayiw1239 & Bonnemaison, J. (1972). Système de grades et diff$\backslash$'\{e\}rences r$\backslash$'\{e\}gionales en Aoba (Nouvelles H$\backslash$'\{e\}brides). Cahiers ORSTOM. S$\backslash$'\{e\}rie Sciences Humaines, 9(1), 87-108.; Tonkinson R (1981) Church and Kastom in Southeast Ambrym. Vanuatu: Politics, Economics and Ritual in Island Melanesia, ed Allen M (Academic Press, Sydney, Australia), pp 237-267.; Humphreys CB (1926) Southern New Hebrides: An Ethnological Record (Cambridge Univ Press, Cambridge, UK); Spriggs M (1982) Taro Cropping Systems in the Southeast Asian-Pacific Region: Archaeological Evidence. Archaeol Ocean 17(1):7-15; Spriggs M (1986) Landscape, Land Use, and Political Transformation in Southern Melanesia. Island Societies: Archaeological Approaches to Evolution and Transformation, ed Kirch PV (Cambridge Univ Press, New York, NY), pp 6-19.; Feinberg R (1988) Socio-Spatial Symbolism and the Logic of Rank on Two Polynesian Outliers. Ethnology 27(3):291-310; Feinberg R (1991) Anuta. Encyclopaedia of World Cultures (Vol II: Oceania) (G.K. Hall and Co, New York, NY), pp 13-16.; Kirch PV (2002) Te Kai Paka-Anuta: Food in a Polynesian Outlier Society. Le Journal de la Soci$\backslash$'\{e\}t$\backslash$'\{e\} des Oc$\backslash$'\{e\}anistes 114-115:71-89.; Bonnemaison, J (1996) The Art of Power. In Bonnemaison (eds) Arts of Vanuatu. University of Hawaii Press.; Davenport WH (1969) Social organization notes on the Northern Santa Cruz Islands: the Main Reef Islands. Baessler-Archiv, Neue Folge 17(1):151-243.; Facey EE (1981) Hereditary chiefship in Nguna. Vanuatu: Politics, Economics and Ritual in Island Melanesia, ed Allen M (Academic Press, Sydney, Australia), pp 295-314. Facey EE (1991) Nguna. Encyclopaedia of World Cultures (Vol II: Oceania) (G.K. Hall and Co, New York, NY), pp 242-244.; Humphreys CB (1926) Southern New Hebrides: An Ethnological Record (Cambridge Univ Press, Cambridge, UK). Spriggs M, Wickler S (1989) Archaeological Research on Erromango: Recent Data on Southern Melanesian Prehistory. Bulletin of the Indo-Pacific Prehistory Association 9:68-91.; Capell A (1958) Culture and Language of Futuna and Aniwa, New Hebrides (Univ Sydney, Sydney, Australia).; Bonnemaison, J (1996) The Art of Power. In Bonnemaison (eds) Arts of Vanuatu. University of Hawaii Press.; Bonnemaison, J (1996) The Art of Power. In Bonnemaison (eds) Arts of Vanuatu. University of Hawaii Press.; Bonnemaison, J (1996) The Art of Power. In Bonnemaison (eds) Arts of Vanuatu. University of Hawaii Press.; Bonnemaison, J (1996) The Art of Power. In Bonnemaison (eds) Arts of Vanuatu. University of Hawaii Press.; Deacon, A. B. 1934. Malekula.; Bonnemaison, J (1996) The Art of Power. In Bonnemaison (eds) Arts of Vanuatu. University of Hawaii Press.; Bonnemaison, J (1996) The Art of Power. In Bonnemaison (eds) Arts of Vanuatu. University of Hawaii Press.; Bonnemaison, J (1996) The Art of Power. In Bonnemaison (eds) Arts of Vanuatu. University of Hawaii Press.; Lane, R. B. 1956. The Heathen Communities of Southeast Pentecost. Journal de la Soci‚te des Oceanistes 12. 139-180., Lane, R. B. 1965. The Melanesians of South Pentecost. In P. Lawrence and M. G. Meggitt (eds.), Gods, Ghosts and Men in Melanesia, 250-279. Lane, R. B., and B. S. Lane. 1957. Unpublished field notes.; Bonnemaison, J (1996) The Art of Power. In Bonnemaison (eds) Arts of Vanuatu. University of Hawaii Press.; Lindström, Lamont (1991) Ajie. Encyclopaedia of World Cultures (Vol II: Oceania) (G.K. Hall and Co, New York, NY), pp 314.; Kirch PV (1994) The Wet and the Dry: Irrigation and Agricultural Intensification in Polynesia (Univ Chicago Press, Chicago, IL). Sahlins MD (1958) Social Stratification in Polynesia (Univ Washington Press, Seattle, WA). Firth R (1939) Primitive Polynesian Economy (George Routledge and Sons, London, UK). Firth R (1959) Social Change in Tikopia: Re-Study of a Polynesian Community after a Generation (Allen and Unwin, London, UK). Firth R (1991) Tikopia. Encyclopaedia of World Cultures (Vol II: Oceania) (G.K. Hall and Co, New York, NY), pp 324-327.; Bonnemaison, J (1996) The Art of Power. In Bonnemaison (eds) Arts of Vanuatu. University of Hawaii Press.; Bonnemaison, J (1996) The Art of Power. In Bonnemaison (eds) Arts of Vanuatu. University of Hawaii Press. \\ 
  Vanuatu and Temotu & Efate & 2.000 & nort2836, nort2836, nort2836 & Bonnemaison, J. (1972). Système de grades et diff$\backslash$'\{e\}rences r$\backslash$'\{e\}gionales en Aoba (Nouvelles H$\backslash$'\{e\}brides). Cahiers ORSTOM. S$\backslash$'\{e\}rie Sciences Humaines, 9(1), 87-108.; Tonkinson R (1981) Church and Kastom in Southeast Ambrym. Vanuatu: Politics, Economics and Ritual in Island Melanesia, ed Allen M (Academic Press, Sydney, Australia), pp 237-267.; Humphreys CB (1926) Southern New Hebrides: An Ethnological Record (Cambridge Univ Press, Cambridge, UK); Spriggs M (1982) Taro Cropping Systems in the Southeast Asian-Pacific Region: Archaeological Evidence. Archaeol Ocean 17(1):7-15; Spriggs M (1986) Landscape, Land Use, and Political Transformation in Southern Melanesia. Island Societies: Archaeological Approaches to Evolution and Transformation, ed Kirch PV (Cambridge Univ Press, New York, NY), pp 6-19.; Feinberg R (1988) Socio-Spatial Symbolism and the Logic of Rank on Two Polynesian Outliers. Ethnology 27(3):291-310; Feinberg R (1991) Anuta. Encyclopaedia of World Cultures (Vol II: Oceania) (G.K. Hall and Co, New York, NY), pp 13-16.; Kirch PV (2002) Te Kai Paka-Anuta: Food in a Polynesian Outlier Society. Le Journal de la Soci$\backslash$'\{e\}t$\backslash$'\{e\} des Oc$\backslash$'\{e\}anistes 114-115:71-89.; Bonnemaison, J (1996) The Art of Power. In Bonnemaison (eds) Arts of Vanuatu. University of Hawaii Press.; Davenport WH (1969) Social organization notes on the Northern Santa Cruz Islands: the Main Reef Islands. Baessler-Archiv, Neue Folge 17(1):151-243.; Facey EE (1981) Hereditary chiefship in Nguna. Vanuatu: Politics, Economics and Ritual in Island Melanesia, ed Allen M (Academic Press, Sydney, Australia), pp 295-314. Facey EE (1991) Nguna. Encyclopaedia of World Cultures (Vol II: Oceania) (G.K. Hall and Co, New York, NY), pp 242-244.; Humphreys CB (1926) Southern New Hebrides: An Ethnological Record (Cambridge Univ Press, Cambridge, UK). Spriggs M, Wickler S (1989) Archaeological Research on Erromango: Recent Data on Southern Melanesian Prehistory. Bulletin of the Indo-Pacific Prehistory Association 9:68-91.; Capell A (1958) Culture and Language of Futuna and Aniwa, New Hebrides (Univ Sydney, Sydney, Australia).; Bonnemaison, J (1996) The Art of Power. In Bonnemaison (eds) Arts of Vanuatu. University of Hawaii Press.; Bonnemaison, J (1996) The Art of Power. In Bonnemaison (eds) Arts of Vanuatu. University of Hawaii Press.; Bonnemaison, J (1996) The Art of Power. In Bonnemaison (eds) Arts of Vanuatu. University of Hawaii Press.; Bonnemaison, J (1996) The Art of Power. In Bonnemaison (eds) Arts of Vanuatu. University of Hawaii Press.; Deacon, A. B. 1934. Malekula.; Bonnemaison, J (1996) The Art of Power. In Bonnemaison (eds) Arts of Vanuatu. University of Hawaii Press.; Bonnemaison, J (1996) The Art of Power. In Bonnemaison (eds) Arts of Vanuatu. University of Hawaii Press.; Bonnemaison, J (1996) The Art of Power. In Bonnemaison (eds) Arts of Vanuatu. University of Hawaii Press.; Lane, R. B. 1956. The Heathen Communities of Southeast Pentecost. Journal de la Soci‚te des Oceanistes 12. 139-180., Lane, R. B. 1965. The Melanesians of South Pentecost. In P. Lawrence and M. G. Meggitt (eds.), Gods, Ghosts and Men in Melanesia, 250-279. Lane, R. B., and B. S. Lane. 1957. Unpublished field notes.; Bonnemaison, J (1996) The Art of Power. In Bonnemaison (eds) Arts of Vanuatu. University of Hawaii Press.; Lindström, Lamont (1991) Ajie. Encyclopaedia of World Cultures (Vol II: Oceania) (G.K. Hall and Co, New York, NY), pp 314.; Kirch PV (1994) The Wet and the Dry: Irrigation and Agricultural Intensification in Polynesia (Univ Chicago Press, Chicago, IL). Sahlins MD (1958) Social Stratification in Polynesia (Univ Washington Press, Seattle, WA). Firth R (1939) Primitive Polynesian Economy (George Routledge and Sons, London, UK). Firth R (1959) Social Change in Tikopia: Re-Study of a Polynesian Community after a Generation (Allen and Unwin, London, UK). Firth R (1991) Tikopia. Encyclopaedia of World Cultures (Vol II: Oceania) (G.K. Hall and Co, New York, NY), pp 324-327.; Bonnemaison, J (1996) The Art of Power. In Bonnemaison (eds) Arts of Vanuatu. University of Hawaii Press.; Bonnemaison, J (1996) The Art of Power. In Bonnemaison (eds) Arts of Vanuatu. University of Hawaii Press. \\ 
  Vanuatu and Temotu & Erromango & 2.000 & siee1239 & Bonnemaison, J. (1972). Système de grades et diff$\backslash$'\{e\}rences r$\backslash$'\{e\}gionales en Aoba (Nouvelles H$\backslash$'\{e\}brides). Cahiers ORSTOM. S$\backslash$'\{e\}rie Sciences Humaines, 9(1), 87-108.; Tonkinson R (1981) Church and Kastom in Southeast Ambrym. Vanuatu: Politics, Economics and Ritual in Island Melanesia, ed Allen M (Academic Press, Sydney, Australia), pp 237-267.; Humphreys CB (1926) Southern New Hebrides: An Ethnological Record (Cambridge Univ Press, Cambridge, UK); Spriggs M (1982) Taro Cropping Systems in the Southeast Asian-Pacific Region: Archaeological Evidence. Archaeol Ocean 17(1):7-15; Spriggs M (1986) Landscape, Land Use, and Political Transformation in Southern Melanesia. Island Societies: Archaeological Approaches to Evolution and Transformation, ed Kirch PV (Cambridge Univ Press, New York, NY), pp 6-19.; Feinberg R (1988) Socio-Spatial Symbolism and the Logic of Rank on Two Polynesian Outliers. Ethnology 27(3):291-310; Feinberg R (1991) Anuta. Encyclopaedia of World Cultures (Vol II: Oceania) (G.K. Hall and Co, New York, NY), pp 13-16.; Kirch PV (2002) Te Kai Paka-Anuta: Food in a Polynesian Outlier Society. Le Journal de la Soci$\backslash$'\{e\}t$\backslash$'\{e\} des Oc$\backslash$'\{e\}anistes 114-115:71-89.; Bonnemaison, J (1996) The Art of Power. In Bonnemaison (eds) Arts of Vanuatu. University of Hawaii Press.; Davenport WH (1969) Social organization notes on the Northern Santa Cruz Islands: the Main Reef Islands. Baessler-Archiv, Neue Folge 17(1):151-243.; Facey EE (1981) Hereditary chiefship in Nguna. Vanuatu: Politics, Economics and Ritual in Island Melanesia, ed Allen M (Academic Press, Sydney, Australia), pp 295-314. Facey EE (1991) Nguna. Encyclopaedia of World Cultures (Vol II: Oceania) (G.K. Hall and Co, New York, NY), pp 242-244.; Humphreys CB (1926) Southern New Hebrides: An Ethnological Record (Cambridge Univ Press, Cambridge, UK). Spriggs M, Wickler S (1989) Archaeological Research on Erromango: Recent Data on Southern Melanesian Prehistory. Bulletin of the Indo-Pacific Prehistory Association 9:68-91.; Capell A (1958) Culture and Language of Futuna and Aniwa, New Hebrides (Univ Sydney, Sydney, Australia).; Bonnemaison, J (1996) The Art of Power. In Bonnemaison (eds) Arts of Vanuatu. University of Hawaii Press.; Bonnemaison, J (1996) The Art of Power. In Bonnemaison (eds) Arts of Vanuatu. University of Hawaii Press.; Bonnemaison, J (1996) The Art of Power. In Bonnemaison (eds) Arts of Vanuatu. University of Hawaii Press.; Bonnemaison, J (1996) The Art of Power. In Bonnemaison (eds) Arts of Vanuatu. University of Hawaii Press.; Deacon, A. B. 1934. Malekula.; Bonnemaison, J (1996) The Art of Power. In Bonnemaison (eds) Arts of Vanuatu. University of Hawaii Press.; Bonnemaison, J (1996) The Art of Power. In Bonnemaison (eds) Arts of Vanuatu. University of Hawaii Press.; Bonnemaison, J (1996) The Art of Power. In Bonnemaison (eds) Arts of Vanuatu. University of Hawaii Press.; Lane, R. B. 1956. The Heathen Communities of Southeast Pentecost. Journal de la Soci‚te des Oceanistes 12. 139-180., Lane, R. B. 1965. The Melanesians of South Pentecost. In P. Lawrence and M. G. Meggitt (eds.), Gods, Ghosts and Men in Melanesia, 250-279. Lane, R. B., and B. S. Lane. 1957. Unpublished field notes.; Bonnemaison, J (1996) The Art of Power. In Bonnemaison (eds) Arts of Vanuatu. University of Hawaii Press.; Lindström, Lamont (1991) Ajie. Encyclopaedia of World Cultures (Vol II: Oceania) (G.K. Hall and Co, New York, NY), pp 314.; Kirch PV (1994) The Wet and the Dry: Irrigation and Agricultural Intensification in Polynesia (Univ Chicago Press, Chicago, IL). Sahlins MD (1958) Social Stratification in Polynesia (Univ Washington Press, Seattle, WA). Firth R (1939) Primitive Polynesian Economy (George Routledge and Sons, London, UK). Firth R (1959) Social Change in Tikopia: Re-Study of a Polynesian Community after a Generation (Allen and Unwin, London, UK). Firth R (1991) Tikopia. Encyclopaedia of World Cultures (Vol II: Oceania) (G.K. Hall and Co, New York, NY), pp 324-327.; Bonnemaison, J (1996) The Art of Power. In Bonnemaison (eds) Arts of Vanuatu. University of Hawaii Press.; Bonnemaison, J (1996) The Art of Power. In Bonnemaison (eds) Arts of Vanuatu. University of Hawaii Press. \\ 
  Vanuatu and Temotu & Futuna and Aniwa & 2.000 & futu1245 & Bonnemaison, J. (1972). Système de grades et diff$\backslash$'\{e\}rences r$\backslash$'\{e\}gionales en Aoba (Nouvelles H$\backslash$'\{e\}brides). Cahiers ORSTOM. S$\backslash$'\{e\}rie Sciences Humaines, 9(1), 87-108.; Tonkinson R (1981) Church and Kastom in Southeast Ambrym. Vanuatu: Politics, Economics and Ritual in Island Melanesia, ed Allen M (Academic Press, Sydney, Australia), pp 237-267.; Humphreys CB (1926) Southern New Hebrides: An Ethnological Record (Cambridge Univ Press, Cambridge, UK); Spriggs M (1982) Taro Cropping Systems in the Southeast Asian-Pacific Region: Archaeological Evidence. Archaeol Ocean 17(1):7-15; Spriggs M (1986) Landscape, Land Use, and Political Transformation in Southern Melanesia. Island Societies: Archaeological Approaches to Evolution and Transformation, ed Kirch PV (Cambridge Univ Press, New York, NY), pp 6-19.; Feinberg R (1988) Socio-Spatial Symbolism and the Logic of Rank on Two Polynesian Outliers. Ethnology 27(3):291-310; Feinberg R (1991) Anuta. Encyclopaedia of World Cultures (Vol II: Oceania) (G.K. Hall and Co, New York, NY), pp 13-16.; Kirch PV (2002) Te Kai Paka-Anuta: Food in a Polynesian Outlier Society. Le Journal de la Soci$\backslash$'\{e\}t$\backslash$'\{e\} des Oc$\backslash$'\{e\}anistes 114-115:71-89.; Bonnemaison, J (1996) The Art of Power. In Bonnemaison (eds) Arts of Vanuatu. University of Hawaii Press.; Davenport WH (1969) Social organization notes on the Northern Santa Cruz Islands: the Main Reef Islands. Baessler-Archiv, Neue Folge 17(1):151-243.; Facey EE (1981) Hereditary chiefship in Nguna. Vanuatu: Politics, Economics and Ritual in Island Melanesia, ed Allen M (Academic Press, Sydney, Australia), pp 295-314. Facey EE (1991) Nguna. Encyclopaedia of World Cultures (Vol II: Oceania) (G.K. Hall and Co, New York, NY), pp 242-244.; Humphreys CB (1926) Southern New Hebrides: An Ethnological Record (Cambridge Univ Press, Cambridge, UK). Spriggs M, Wickler S (1989) Archaeological Research on Erromango: Recent Data on Southern Melanesian Prehistory. Bulletin of the Indo-Pacific Prehistory Association 9:68-91.; Capell A (1958) Culture and Language of Futuna and Aniwa, New Hebrides (Univ Sydney, Sydney, Australia).; Bonnemaison, J (1996) The Art of Power. In Bonnemaison (eds) Arts of Vanuatu. University of Hawaii Press.; Bonnemaison, J (1996) The Art of Power. In Bonnemaison (eds) Arts of Vanuatu. University of Hawaii Press.; Bonnemaison, J (1996) The Art of Power. In Bonnemaison (eds) Arts of Vanuatu. University of Hawaii Press.; Bonnemaison, J (1996) The Art of Power. In Bonnemaison (eds) Arts of Vanuatu. University of Hawaii Press.; Deacon, A. B. 1934. Malekula.; Bonnemaison, J (1996) The Art of Power. In Bonnemaison (eds) Arts of Vanuatu. University of Hawaii Press.; Bonnemaison, J (1996) The Art of Power. In Bonnemaison (eds) Arts of Vanuatu. University of Hawaii Press.; Bonnemaison, J (1996) The Art of Power. In Bonnemaison (eds) Arts of Vanuatu. University of Hawaii Press.; Lane, R. B. 1956. The Heathen Communities of Southeast Pentecost. Journal de la Soci‚te des Oceanistes 12. 139-180., Lane, R. B. 1965. The Melanesians of South Pentecost. In P. Lawrence and M. G. Meggitt (eds.), Gods, Ghosts and Men in Melanesia, 250-279. Lane, R. B., and B. S. Lane. 1957. Unpublished field notes.; Bonnemaison, J (1996) The Art of Power. In Bonnemaison (eds) Arts of Vanuatu. University of Hawaii Press.; Lindström, Lamont (1991) Ajie. Encyclopaedia of World Cultures (Vol II: Oceania) (G.K. Hall and Co, New York, NY), pp 314.; Kirch PV (1994) The Wet and the Dry: Irrigation and Agricultural Intensification in Polynesia (Univ Chicago Press, Chicago, IL). Sahlins MD (1958) Social Stratification in Polynesia (Univ Washington Press, Seattle, WA). Firth R (1939) Primitive Polynesian Economy (George Routledge and Sons, London, UK). Firth R (1959) Social Change in Tikopia: Re-Study of a Polynesian Community after a Generation (Allen and Unwin, London, UK). Firth R (1991) Tikopia. Encyclopaedia of World Cultures (Vol II: Oceania) (G.K. Hall and Co, New York, NY), pp 324-327.; Bonnemaison, J (1996) The Art of Power. In Bonnemaison (eds) Arts of Vanuatu. University of Hawaii Press.; Bonnemaison, J (1996) The Art of Power. In Bonnemaison (eds) Arts of Vanuatu. University of Hawaii Press. \\ 
  Vanuatu and Temotu & Hiu & 1.000 & hiww1237 & Bonnemaison, J. (1972). Système de grades et diff$\backslash$'\{e\}rences r$\backslash$'\{e\}gionales en Aoba (Nouvelles H$\backslash$'\{e\}brides). Cahiers ORSTOM. S$\backslash$'\{e\}rie Sciences Humaines, 9(1), 87-108.; Tonkinson R (1981) Church and Kastom in Southeast Ambrym. Vanuatu: Politics, Economics and Ritual in Island Melanesia, ed Allen M (Academic Press, Sydney, Australia), pp 237-267.; Humphreys CB (1926) Southern New Hebrides: An Ethnological Record (Cambridge Univ Press, Cambridge, UK); Spriggs M (1982) Taro Cropping Systems in the Southeast Asian-Pacific Region: Archaeological Evidence. Archaeol Ocean 17(1):7-15; Spriggs M (1986) Landscape, Land Use, and Political Transformation in Southern Melanesia. Island Societies: Archaeological Approaches to Evolution and Transformation, ed Kirch PV (Cambridge Univ Press, New York, NY), pp 6-19.; Feinberg R (1988) Socio-Spatial Symbolism and the Logic of Rank on Two Polynesian Outliers. Ethnology 27(3):291-310; Feinberg R (1991) Anuta. Encyclopaedia of World Cultures (Vol II: Oceania) (G.K. Hall and Co, New York, NY), pp 13-16.; Kirch PV (2002) Te Kai Paka-Anuta: Food in a Polynesian Outlier Society. Le Journal de la Soci$\backslash$'\{e\}t$\backslash$'\{e\} des Oc$\backslash$'\{e\}anistes 114-115:71-89.; Bonnemaison, J (1996) The Art of Power. In Bonnemaison (eds) Arts of Vanuatu. University of Hawaii Press.; Davenport WH (1969) Social organization notes on the Northern Santa Cruz Islands: the Main Reef Islands. Baessler-Archiv, Neue Folge 17(1):151-243.; Facey EE (1981) Hereditary chiefship in Nguna. Vanuatu: Politics, Economics and Ritual in Island Melanesia, ed Allen M (Academic Press, Sydney, Australia), pp 295-314. Facey EE (1991) Nguna. Encyclopaedia of World Cultures (Vol II: Oceania) (G.K. Hall and Co, New York, NY), pp 242-244.; Humphreys CB (1926) Southern New Hebrides: An Ethnological Record (Cambridge Univ Press, Cambridge, UK). Spriggs M, Wickler S (1989) Archaeological Research on Erromango: Recent Data on Southern Melanesian Prehistory. Bulletin of the Indo-Pacific Prehistory Association 9:68-91.; Capell A (1958) Culture and Language of Futuna and Aniwa, New Hebrides (Univ Sydney, Sydney, Australia).; Bonnemaison, J (1996) The Art of Power. In Bonnemaison (eds) Arts of Vanuatu. University of Hawaii Press.; Bonnemaison, J (1996) The Art of Power. In Bonnemaison (eds) Arts of Vanuatu. University of Hawaii Press.; Bonnemaison, J (1996) The Art of Power. In Bonnemaison (eds) Arts of Vanuatu. University of Hawaii Press.; Bonnemaison, J (1996) The Art of Power. In Bonnemaison (eds) Arts of Vanuatu. University of Hawaii Press.; Deacon, A. B. 1934. Malekula.; Bonnemaison, J (1996) The Art of Power. In Bonnemaison (eds) Arts of Vanuatu. University of Hawaii Press.; Bonnemaison, J (1996) The Art of Power. In Bonnemaison (eds) Arts of Vanuatu. University of Hawaii Press.; Bonnemaison, J (1996) The Art of Power. In Bonnemaison (eds) Arts of Vanuatu. University of Hawaii Press.; Lane, R. B. 1956. The Heathen Communities of Southeast Pentecost. Journal de la Soci‚te des Oceanistes 12. 139-180., Lane, R. B. 1965. The Melanesians of South Pentecost. In P. Lawrence and M. G. Meggitt (eds.), Gods, Ghosts and Men in Melanesia, 250-279. Lane, R. B., and B. S. Lane. 1957. Unpublished field notes.; Bonnemaison, J (1996) The Art of Power. In Bonnemaison (eds) Arts of Vanuatu. University of Hawaii Press.; Lindström, Lamont (1991) Ajie. Encyclopaedia of World Cultures (Vol II: Oceania) (G.K. Hall and Co, New York, NY), pp 314.; Kirch PV (1994) The Wet and the Dry: Irrigation and Agricultural Intensification in Polynesia (Univ Chicago Press, Chicago, IL). Sahlins MD (1958) Social Stratification in Polynesia (Univ Washington Press, Seattle, WA). Firth R (1939) Primitive Polynesian Economy (George Routledge and Sons, London, UK). Firth R (1959) Social Change in Tikopia: Re-Study of a Polynesian Community after a Generation (Allen and Unwin, London, UK). Firth R (1991) Tikopia. Encyclopaedia of World Cultures (Vol II: Oceania) (G.K. Hall and Co, New York, NY), pp 324-327.; Bonnemaison, J (1996) The Art of Power. In Bonnemaison (eds) Arts of Vanuatu. University of Hawaii Press.; Bonnemaison, J (1996) The Art of Power. In Bonnemaison (eds) Arts of Vanuatu. University of Hawaii Press. \\ 
  Vanuatu and Temotu & Loh-Toga & 1.000 & loto1240, loto1240 & Bonnemaison, J. (1972). Système de grades et diff$\backslash$'\{e\}rences r$\backslash$'\{e\}gionales en Aoba (Nouvelles H$\backslash$'\{e\}brides). Cahiers ORSTOM. S$\backslash$'\{e\}rie Sciences Humaines, 9(1), 87-108.; Tonkinson R (1981) Church and Kastom in Southeast Ambrym. Vanuatu: Politics, Economics and Ritual in Island Melanesia, ed Allen M (Academic Press, Sydney, Australia), pp 237-267.; Humphreys CB (1926) Southern New Hebrides: An Ethnological Record (Cambridge Univ Press, Cambridge, UK); Spriggs M (1982) Taro Cropping Systems in the Southeast Asian-Pacific Region: Archaeological Evidence. Archaeol Ocean 17(1):7-15; Spriggs M (1986) Landscape, Land Use, and Political Transformation in Southern Melanesia. Island Societies: Archaeological Approaches to Evolution and Transformation, ed Kirch PV (Cambridge Univ Press, New York, NY), pp 6-19.; Feinberg R (1988) Socio-Spatial Symbolism and the Logic of Rank on Two Polynesian Outliers. Ethnology 27(3):291-310; Feinberg R (1991) Anuta. Encyclopaedia of World Cultures (Vol II: Oceania) (G.K. Hall and Co, New York, NY), pp 13-16.; Kirch PV (2002) Te Kai Paka-Anuta: Food in a Polynesian Outlier Society. Le Journal de la Soci$\backslash$'\{e\}t$\backslash$'\{e\} des Oc$\backslash$'\{e\}anistes 114-115:71-89.; Bonnemaison, J (1996) The Art of Power. In Bonnemaison (eds) Arts of Vanuatu. University of Hawaii Press.; Davenport WH (1969) Social organization notes on the Northern Santa Cruz Islands: the Main Reef Islands. Baessler-Archiv, Neue Folge 17(1):151-243.; Facey EE (1981) Hereditary chiefship in Nguna. Vanuatu: Politics, Economics and Ritual in Island Melanesia, ed Allen M (Academic Press, Sydney, Australia), pp 295-314. Facey EE (1991) Nguna. Encyclopaedia of World Cultures (Vol II: Oceania) (G.K. Hall and Co, New York, NY), pp 242-244.; Humphreys CB (1926) Southern New Hebrides: An Ethnological Record (Cambridge Univ Press, Cambridge, UK). Spriggs M, Wickler S (1989) Archaeological Research on Erromango: Recent Data on Southern Melanesian Prehistory. Bulletin of the Indo-Pacific Prehistory Association 9:68-91.; Capell A (1958) Culture and Language of Futuna and Aniwa, New Hebrides (Univ Sydney, Sydney, Australia).; Bonnemaison, J (1996) The Art of Power. In Bonnemaison (eds) Arts of Vanuatu. University of Hawaii Press.; Bonnemaison, J (1996) The Art of Power. In Bonnemaison (eds) Arts of Vanuatu. University of Hawaii Press.; Bonnemaison, J (1996) The Art of Power. In Bonnemaison (eds) Arts of Vanuatu. University of Hawaii Press.; Bonnemaison, J (1996) The Art of Power. In Bonnemaison (eds) Arts of Vanuatu. University of Hawaii Press.; Deacon, A. B. 1934. Malekula.; Bonnemaison, J (1996) The Art of Power. In Bonnemaison (eds) Arts of Vanuatu. University of Hawaii Press.; Bonnemaison, J (1996) The Art of Power. In Bonnemaison (eds) Arts of Vanuatu. University of Hawaii Press.; Bonnemaison, J (1996) The Art of Power. In Bonnemaison (eds) Arts of Vanuatu. University of Hawaii Press.; Lane, R. B. 1956. The Heathen Communities of Southeast Pentecost. Journal de la Soci‚te des Oceanistes 12. 139-180., Lane, R. B. 1965. The Melanesians of South Pentecost. In P. Lawrence and M. G. Meggitt (eds.), Gods, Ghosts and Men in Melanesia, 250-279. Lane, R. B., and B. S. Lane. 1957. Unpublished field notes.; Bonnemaison, J (1996) The Art of Power. In Bonnemaison (eds) Arts of Vanuatu. University of Hawaii Press.; Lindström, Lamont (1991) Ajie. Encyclopaedia of World Cultures (Vol II: Oceania) (G.K. Hall and Co, New York, NY), pp 314.; Kirch PV (1994) The Wet and the Dry: Irrigation and Agricultural Intensification in Polynesia (Univ Chicago Press, Chicago, IL). Sahlins MD (1958) Social Stratification in Polynesia (Univ Washington Press, Seattle, WA). Firth R (1939) Primitive Polynesian Economy (George Routledge and Sons, London, UK). Firth R (1959) Social Change in Tikopia: Re-Study of a Polynesian Community after a Generation (Allen and Unwin, London, UK). Firth R (1991) Tikopia. Encyclopaedia of World Cultures (Vol II: Oceania) (G.K. Hall and Co, New York, NY), pp 324-327.; Bonnemaison, J (1996) The Art of Power. In Bonnemaison (eds) Arts of Vanuatu. University of Hawaii Press.; Bonnemaison, J (1996) The Art of Power. In Bonnemaison (eds) Arts of Vanuatu. University of Hawaii Press. \\ 
  Vanuatu and Temotu & Maewo & 1.000 & baet1237, cent2058, mari1426 & Bonnemaison, J. (1972). Système de grades et diff$\backslash$'\{e\}rences r$\backslash$'\{e\}gionales en Aoba (Nouvelles H$\backslash$'\{e\}brides). Cahiers ORSTOM. S$\backslash$'\{e\}rie Sciences Humaines, 9(1), 87-108.; Tonkinson R (1981) Church and Kastom in Southeast Ambrym. Vanuatu: Politics, Economics and Ritual in Island Melanesia, ed Allen M (Academic Press, Sydney, Australia), pp 237-267.; Humphreys CB (1926) Southern New Hebrides: An Ethnological Record (Cambridge Univ Press, Cambridge, UK); Spriggs M (1982) Taro Cropping Systems in the Southeast Asian-Pacific Region: Archaeological Evidence. Archaeol Ocean 17(1):7-15; Spriggs M (1986) Landscape, Land Use, and Political Transformation in Southern Melanesia. Island Societies: Archaeological Approaches to Evolution and Transformation, ed Kirch PV (Cambridge Univ Press, New York, NY), pp 6-19.; Feinberg R (1988) Socio-Spatial Symbolism and the Logic of Rank on Two Polynesian Outliers. Ethnology 27(3):291-310; Feinberg R (1991) Anuta. Encyclopaedia of World Cultures (Vol II: Oceania) (G.K. Hall and Co, New York, NY), pp 13-16.; Kirch PV (2002) Te Kai Paka-Anuta: Food in a Polynesian Outlier Society. Le Journal de la Soci$\backslash$'\{e\}t$\backslash$'\{e\} des Oc$\backslash$'\{e\}anistes 114-115:71-89.; Bonnemaison, J (1996) The Art of Power. In Bonnemaison (eds) Arts of Vanuatu. University of Hawaii Press.; Davenport WH (1969) Social organization notes on the Northern Santa Cruz Islands: the Main Reef Islands. Baessler-Archiv, Neue Folge 17(1):151-243.; Facey EE (1981) Hereditary chiefship in Nguna. Vanuatu: Politics, Economics and Ritual in Island Melanesia, ed Allen M (Academic Press, Sydney, Australia), pp 295-314. Facey EE (1991) Nguna. Encyclopaedia of World Cultures (Vol II: Oceania) (G.K. Hall and Co, New York, NY), pp 242-244.; Humphreys CB (1926) Southern New Hebrides: An Ethnological Record (Cambridge Univ Press, Cambridge, UK). Spriggs M, Wickler S (1989) Archaeological Research on Erromango: Recent Data on Southern Melanesian Prehistory. Bulletin of the Indo-Pacific Prehistory Association 9:68-91.; Capell A (1958) Culture and Language of Futuna and Aniwa, New Hebrides (Univ Sydney, Sydney, Australia).; Bonnemaison, J (1996) The Art of Power. In Bonnemaison (eds) Arts of Vanuatu. University of Hawaii Press.; Bonnemaison, J (1996) The Art of Power. In Bonnemaison (eds) Arts of Vanuatu. University of Hawaii Press.; Bonnemaison, J (1996) The Art of Power. In Bonnemaison (eds) Arts of Vanuatu. University of Hawaii Press.; Bonnemaison, J (1996) The Art of Power. In Bonnemaison (eds) Arts of Vanuatu. University of Hawaii Press.; Deacon, A. B. 1934. Malekula.; Bonnemaison, J (1996) The Art of Power. In Bonnemaison (eds) Arts of Vanuatu. University of Hawaii Press.; Bonnemaison, J (1996) The Art of Power. In Bonnemaison (eds) Arts of Vanuatu. University of Hawaii Press.; Bonnemaison, J (1996) The Art of Power. In Bonnemaison (eds) Arts of Vanuatu. University of Hawaii Press.; Lane, R. B. 1956. The Heathen Communities of Southeast Pentecost. Journal de la Soci‚te des Oceanistes 12. 139-180., Lane, R. B. 1965. The Melanesians of South Pentecost. In P. Lawrence and M. G. Meggitt (eds.), Gods, Ghosts and Men in Melanesia, 250-279. Lane, R. B., and B. S. Lane. 1957. Unpublished field notes.; Bonnemaison, J (1996) The Art of Power. In Bonnemaison (eds) Arts of Vanuatu. University of Hawaii Press.; Lindström, Lamont (1991) Ajie. Encyclopaedia of World Cultures (Vol II: Oceania) (G.K. Hall and Co, New York, NY), pp 314.; Kirch PV (1994) The Wet and the Dry: Irrigation and Agricultural Intensification in Polynesia (Univ Chicago Press, Chicago, IL). Sahlins MD (1958) Social Stratification in Polynesia (Univ Washington Press, Seattle, WA). Firth R (1939) Primitive Polynesian Economy (George Routledge and Sons, London, UK). Firth R (1959) Social Change in Tikopia: Re-Study of a Polynesian Community after a Generation (Allen and Unwin, London, UK). Firth R (1991) Tikopia. Encyclopaedia of World Cultures (Vol II: Oceania) (G.K. Hall and Co, New York, NY), pp 324-327.; Bonnemaison, J (1996) The Art of Power. In Bonnemaison (eds) Arts of Vanuatu. University of Hawaii Press.; Bonnemaison, J (1996) The Art of Power. In Bonnemaison (eds) Arts of Vanuatu. University of Hawaii Press. \\ 
  Vanuatu and Temotu & Malakula & 1.000 & aulu1238, avok1244, axam1237, bign1238, burm1263, bwen1239, dixo1238, katb1237, labo1244, lare1249, lete1241, ling1265, litz1237, maee1241, malf1237, malu1245, mara1399, mask1242, mpot1241, nese1235, nisv1234, niti1249, port1285, rere1240, unua1237, urip1239, vinm1237, vivt1234, sout2857 & Bonnemaison, J. (1972). Système de grades et diff$\backslash$'\{e\}rences r$\backslash$'\{e\}gionales en Aoba (Nouvelles H$\backslash$'\{e\}brides). Cahiers ORSTOM. S$\backslash$'\{e\}rie Sciences Humaines, 9(1), 87-108.; Tonkinson R (1981) Church and Kastom in Southeast Ambrym. Vanuatu: Politics, Economics and Ritual in Island Melanesia, ed Allen M (Academic Press, Sydney, Australia), pp 237-267.; Humphreys CB (1926) Southern New Hebrides: An Ethnological Record (Cambridge Univ Press, Cambridge, UK); Spriggs M (1982) Taro Cropping Systems in the Southeast Asian-Pacific Region: Archaeological Evidence. Archaeol Ocean 17(1):7-15; Spriggs M (1986) Landscape, Land Use, and Political Transformation in Southern Melanesia. Island Societies: Archaeological Approaches to Evolution and Transformation, ed Kirch PV (Cambridge Univ Press, New York, NY), pp 6-19.; Feinberg R (1988) Socio-Spatial Symbolism and the Logic of Rank on Two Polynesian Outliers. Ethnology 27(3):291-310; Feinberg R (1991) Anuta. Encyclopaedia of World Cultures (Vol II: Oceania) (G.K. Hall and Co, New York, NY), pp 13-16.; Kirch PV (2002) Te Kai Paka-Anuta: Food in a Polynesian Outlier Society. Le Journal de la Soci$\backslash$'\{e\}t$\backslash$'\{e\} des Oc$\backslash$'\{e\}anistes 114-115:71-89.; Bonnemaison, J (1996) The Art of Power. In Bonnemaison (eds) Arts of Vanuatu. University of Hawaii Press.; Davenport WH (1969) Social organization notes on the Northern Santa Cruz Islands: the Main Reef Islands. Baessler-Archiv, Neue Folge 17(1):151-243.; Facey EE (1981) Hereditary chiefship in Nguna. Vanuatu: Politics, Economics and Ritual in Island Melanesia, ed Allen M (Academic Press, Sydney, Australia), pp 295-314. Facey EE (1991) Nguna. Encyclopaedia of World Cultures (Vol II: Oceania) (G.K. Hall and Co, New York, NY), pp 242-244.; Humphreys CB (1926) Southern New Hebrides: An Ethnological Record (Cambridge Univ Press, Cambridge, UK). Spriggs M, Wickler S (1989) Archaeological Research on Erromango: Recent Data on Southern Melanesian Prehistory. Bulletin of the Indo-Pacific Prehistory Association 9:68-91.; Capell A (1958) Culture and Language of Futuna and Aniwa, New Hebrides (Univ Sydney, Sydney, Australia).; Bonnemaison, J (1996) The Art of Power. In Bonnemaison (eds) Arts of Vanuatu. University of Hawaii Press.; Bonnemaison, J (1996) The Art of Power. In Bonnemaison (eds) Arts of Vanuatu. University of Hawaii Press.; Bonnemaison, J (1996) The Art of Power. In Bonnemaison (eds) Arts of Vanuatu. University of Hawaii Press.; Bonnemaison, J (1996) The Art of Power. In Bonnemaison (eds) Arts of Vanuatu. University of Hawaii Press.; Deacon, A. B. 1934. Malekula.; Bonnemaison, J (1996) The Art of Power. In Bonnemaison (eds) Arts of Vanuatu. University of Hawaii Press.; Bonnemaison, J (1996) The Art of Power. In Bonnemaison (eds) Arts of Vanuatu. University of Hawaii Press.; Bonnemaison, J (1996) The Art of Power. In Bonnemaison (eds) Arts of Vanuatu. University of Hawaii Press.; Lane, R. B. 1956. The Heathen Communities of Southeast Pentecost. Journal de la Soci‚te des Oceanistes 12. 139-180., Lane, R. B. 1965. The Melanesians of South Pentecost. In P. Lawrence and M. G. Meggitt (eds.), Gods, Ghosts and Men in Melanesia, 250-279. Lane, R. B., and B. S. Lane. 1957. Unpublished field notes.; Bonnemaison, J (1996) The Art of Power. In Bonnemaison (eds) Arts of Vanuatu. University of Hawaii Press.; Lindström, Lamont (1991) Ajie. Encyclopaedia of World Cultures (Vol II: Oceania) (G.K. Hall and Co, New York, NY), pp 314.; Kirch PV (1994) The Wet and the Dry: Irrigation and Agricultural Intensification in Polynesia (Univ Chicago Press, Chicago, IL). Sahlins MD (1958) Social Stratification in Polynesia (Univ Washington Press, Seattle, WA). Firth R (1939) Primitive Polynesian Economy (George Routledge and Sons, London, UK). Firth R (1959) Social Change in Tikopia: Re-Study of a Polynesian Community after a Generation (Allen and Unwin, London, UK). Firth R (1991) Tikopia. Encyclopaedia of World Cultures (Vol II: Oceania) (G.K. Hall and Co, New York, NY), pp 324-327.; Bonnemaison, J (1996) The Art of Power. In Bonnemaison (eds) Arts of Vanuatu. University of Hawaii Press.; Bonnemaison, J (1996) The Art of Power. In Bonnemaison (eds) Arts of Vanuatu. University of Hawaii Press. \\ 
  Vanuatu and Temotu & Mota & 1.000 & mota1237 & Bonnemaison, J. (1972). Système de grades et diff$\backslash$'\{e\}rences r$\backslash$'\{e\}gionales en Aoba (Nouvelles H$\backslash$'\{e\}brides). Cahiers ORSTOM. S$\backslash$'\{e\}rie Sciences Humaines, 9(1), 87-108.; Tonkinson R (1981) Church and Kastom in Southeast Ambrym. Vanuatu: Politics, Economics and Ritual in Island Melanesia, ed Allen M (Academic Press, Sydney, Australia), pp 237-267.; Humphreys CB (1926) Southern New Hebrides: An Ethnological Record (Cambridge Univ Press, Cambridge, UK); Spriggs M (1982) Taro Cropping Systems in the Southeast Asian-Pacific Region: Archaeological Evidence. Archaeol Ocean 17(1):7-15; Spriggs M (1986) Landscape, Land Use, and Political Transformation in Southern Melanesia. Island Societies: Archaeological Approaches to Evolution and Transformation, ed Kirch PV (Cambridge Univ Press, New York, NY), pp 6-19.; Feinberg R (1988) Socio-Spatial Symbolism and the Logic of Rank on Two Polynesian Outliers. Ethnology 27(3):291-310; Feinberg R (1991) Anuta. Encyclopaedia of World Cultures (Vol II: Oceania) (G.K. Hall and Co, New York, NY), pp 13-16.; Kirch PV (2002) Te Kai Paka-Anuta: Food in a Polynesian Outlier Society. Le Journal de la Soci$\backslash$'\{e\}t$\backslash$'\{e\} des Oc$\backslash$'\{e\}anistes 114-115:71-89.; Bonnemaison, J (1996) The Art of Power. In Bonnemaison (eds) Arts of Vanuatu. University of Hawaii Press.; Davenport WH (1969) Social organization notes on the Northern Santa Cruz Islands: the Main Reef Islands. Baessler-Archiv, Neue Folge 17(1):151-243.; Facey EE (1981) Hereditary chiefship in Nguna. Vanuatu: Politics, Economics and Ritual in Island Melanesia, ed Allen M (Academic Press, Sydney, Australia), pp 295-314. Facey EE (1991) Nguna. Encyclopaedia of World Cultures (Vol II: Oceania) (G.K. Hall and Co, New York, NY), pp 242-244.; Humphreys CB (1926) Southern New Hebrides: An Ethnological Record (Cambridge Univ Press, Cambridge, UK). Spriggs M, Wickler S (1989) Archaeological Research on Erromango: Recent Data on Southern Melanesian Prehistory. Bulletin of the Indo-Pacific Prehistory Association 9:68-91.; Capell A (1958) Culture and Language of Futuna and Aniwa, New Hebrides (Univ Sydney, Sydney, Australia).; Bonnemaison, J (1996) The Art of Power. In Bonnemaison (eds) Arts of Vanuatu. University of Hawaii Press.; Bonnemaison, J (1996) The Art of Power. In Bonnemaison (eds) Arts of Vanuatu. University of Hawaii Press.; Bonnemaison, J (1996) The Art of Power. In Bonnemaison (eds) Arts of Vanuatu. University of Hawaii Press.; Bonnemaison, J (1996) The Art of Power. In Bonnemaison (eds) Arts of Vanuatu. University of Hawaii Press.; Deacon, A. B. 1934. Malekula.; Bonnemaison, J (1996) The Art of Power. In Bonnemaison (eds) Arts of Vanuatu. University of Hawaii Press.; Bonnemaison, J (1996) The Art of Power. In Bonnemaison (eds) Arts of Vanuatu. University of Hawaii Press.; Bonnemaison, J (1996) The Art of Power. In Bonnemaison (eds) Arts of Vanuatu. University of Hawaii Press.; Lane, R. B. 1956. The Heathen Communities of Southeast Pentecost. Journal de la Soci‚te des Oceanistes 12. 139-180., Lane, R. B. 1965. The Melanesians of South Pentecost. In P. Lawrence and M. G. Meggitt (eds.), Gods, Ghosts and Men in Melanesia, 250-279. Lane, R. B., and B. S. Lane. 1957. Unpublished field notes.; Bonnemaison, J (1996) The Art of Power. In Bonnemaison (eds) Arts of Vanuatu. University of Hawaii Press.; Lindström, Lamont (1991) Ajie. Encyclopaedia of World Cultures (Vol II: Oceania) (G.K. Hall and Co, New York, NY), pp 314.; Kirch PV (1994) The Wet and the Dry: Irrigation and Agricultural Intensification in Polynesia (Univ Chicago Press, Chicago, IL). Sahlins MD (1958) Social Stratification in Polynesia (Univ Washington Press, Seattle, WA). Firth R (1939) Primitive Polynesian Economy (George Routledge and Sons, London, UK). Firth R (1959) Social Change in Tikopia: Re-Study of a Polynesian Community after a Generation (Allen and Unwin, London, UK). Firth R (1991) Tikopia. Encyclopaedia of World Cultures (Vol II: Oceania) (G.K. Hall and Co, New York, NY), pp 324-327.; Bonnemaison, J (1996) The Art of Power. In Bonnemaison (eds) Arts of Vanuatu. University of Hawaii Press.; Bonnemaison, J (1996) The Art of Power. In Bonnemaison (eds) Arts of Vanuatu. University of Hawaii Press. \\ 
  Vanuatu and Temotu & Mota Lava & 1.000 & motl1237 & Bonnemaison, J. (1972). Système de grades et diff$\backslash$'\{e\}rences r$\backslash$'\{e\}gionales en Aoba (Nouvelles H$\backslash$'\{e\}brides). Cahiers ORSTOM. S$\backslash$'\{e\}rie Sciences Humaines, 9(1), 87-108.; Tonkinson R (1981) Church and Kastom in Southeast Ambrym. Vanuatu: Politics, Economics and Ritual in Island Melanesia, ed Allen M (Academic Press, Sydney, Australia), pp 237-267.; Humphreys CB (1926) Southern New Hebrides: An Ethnological Record (Cambridge Univ Press, Cambridge, UK); Spriggs M (1982) Taro Cropping Systems in the Southeast Asian-Pacific Region: Archaeological Evidence. Archaeol Ocean 17(1):7-15; Spriggs M (1986) Landscape, Land Use, and Political Transformation in Southern Melanesia. Island Societies: Archaeological Approaches to Evolution and Transformation, ed Kirch PV (Cambridge Univ Press, New York, NY), pp 6-19.; Feinberg R (1988) Socio-Spatial Symbolism and the Logic of Rank on Two Polynesian Outliers. Ethnology 27(3):291-310; Feinberg R (1991) Anuta. Encyclopaedia of World Cultures (Vol II: Oceania) (G.K. Hall and Co, New York, NY), pp 13-16.; Kirch PV (2002) Te Kai Paka-Anuta: Food in a Polynesian Outlier Society. Le Journal de la Soci$\backslash$'\{e\}t$\backslash$'\{e\} des Oc$\backslash$'\{e\}anistes 114-115:71-89.; Bonnemaison, J (1996) The Art of Power. In Bonnemaison (eds) Arts of Vanuatu. University of Hawaii Press.; Davenport WH (1969) Social organization notes on the Northern Santa Cruz Islands: the Main Reef Islands. Baessler-Archiv, Neue Folge 17(1):151-243.; Facey EE (1981) Hereditary chiefship in Nguna. Vanuatu: Politics, Economics and Ritual in Island Melanesia, ed Allen M (Academic Press, Sydney, Australia), pp 295-314. Facey EE (1991) Nguna. Encyclopaedia of World Cultures (Vol II: Oceania) (G.K. Hall and Co, New York, NY), pp 242-244.; Humphreys CB (1926) Southern New Hebrides: An Ethnological Record (Cambridge Univ Press, Cambridge, UK). Spriggs M, Wickler S (1989) Archaeological Research on Erromango: Recent Data on Southern Melanesian Prehistory. Bulletin of the Indo-Pacific Prehistory Association 9:68-91.; Capell A (1958) Culture and Language of Futuna and Aniwa, New Hebrides (Univ Sydney, Sydney, Australia).; Bonnemaison, J (1996) The Art of Power. In Bonnemaison (eds) Arts of Vanuatu. University of Hawaii Press.; Bonnemaison, J (1996) The Art of Power. In Bonnemaison (eds) Arts of Vanuatu. University of Hawaii Press.; Bonnemaison, J (1996) The Art of Power. In Bonnemaison (eds) Arts of Vanuatu. University of Hawaii Press.; Bonnemaison, J (1996) The Art of Power. In Bonnemaison (eds) Arts of Vanuatu. University of Hawaii Press.; Deacon, A. B. 1934. Malekula.; Bonnemaison, J (1996) The Art of Power. In Bonnemaison (eds) Arts of Vanuatu. University of Hawaii Press.; Bonnemaison, J (1996) The Art of Power. In Bonnemaison (eds) Arts of Vanuatu. University of Hawaii Press.; Bonnemaison, J (1996) The Art of Power. In Bonnemaison (eds) Arts of Vanuatu. University of Hawaii Press.; Lane, R. B. 1956. The Heathen Communities of Southeast Pentecost. Journal de la Soci‚te des Oceanistes 12. 139-180., Lane, R. B. 1965. The Melanesians of South Pentecost. In P. Lawrence and M. G. Meggitt (eds.), Gods, Ghosts and Men in Melanesia, 250-279. Lane, R. B., and B. S. Lane. 1957. Unpublished field notes.; Bonnemaison, J (1996) The Art of Power. In Bonnemaison (eds) Arts of Vanuatu. University of Hawaii Press.; Lindström, Lamont (1991) Ajie. Encyclopaedia of World Cultures (Vol II: Oceania) (G.K. Hall and Co, New York, NY), pp 314.; Kirch PV (1994) The Wet and the Dry: Irrigation and Agricultural Intensification in Polynesia (Univ Chicago Press, Chicago, IL). Sahlins MD (1958) Social Stratification in Polynesia (Univ Washington Press, Seattle, WA). Firth R (1939) Primitive Polynesian Economy (George Routledge and Sons, London, UK). Firth R (1959) Social Change in Tikopia: Re-Study of a Polynesian Community after a Generation (Allen and Unwin, London, UK). Firth R (1991) Tikopia. Encyclopaedia of World Cultures (Vol II: Oceania) (G.K. Hall and Co, New York, NY), pp 324-327.; Bonnemaison, J (1996) The Art of Power. In Bonnemaison (eds) Arts of Vanuatu. University of Hawaii Press.; Bonnemaison, J (1996) The Art of Power. In Bonnemaison (eds) Arts of Vanuatu. University of Hawaii Press. \\ 
  Vanuatu and Temotu & Paama & 1.000 & paam1238, paam1238 & Bonnemaison, J. (1972). Système de grades et diff$\backslash$'\{e\}rences r$\backslash$'\{e\}gionales en Aoba (Nouvelles H$\backslash$'\{e\}brides). Cahiers ORSTOM. S$\backslash$'\{e\}rie Sciences Humaines, 9(1), 87-108.; Tonkinson R (1981) Church and Kastom in Southeast Ambrym. Vanuatu: Politics, Economics and Ritual in Island Melanesia, ed Allen M (Academic Press, Sydney, Australia), pp 237-267.; Humphreys CB (1926) Southern New Hebrides: An Ethnological Record (Cambridge Univ Press, Cambridge, UK); Spriggs M (1982) Taro Cropping Systems in the Southeast Asian-Pacific Region: Archaeological Evidence. Archaeol Ocean 17(1):7-15; Spriggs M (1986) Landscape, Land Use, and Political Transformation in Southern Melanesia. Island Societies: Archaeological Approaches to Evolution and Transformation, ed Kirch PV (Cambridge Univ Press, New York, NY), pp 6-19.; Feinberg R (1988) Socio-Spatial Symbolism and the Logic of Rank on Two Polynesian Outliers. Ethnology 27(3):291-310; Feinberg R (1991) Anuta. Encyclopaedia of World Cultures (Vol II: Oceania) (G.K. Hall and Co, New York, NY), pp 13-16.; Kirch PV (2002) Te Kai Paka-Anuta: Food in a Polynesian Outlier Society. Le Journal de la Soci$\backslash$'\{e\}t$\backslash$'\{e\} des Oc$\backslash$'\{e\}anistes 114-115:71-89.; Bonnemaison, J (1996) The Art of Power. In Bonnemaison (eds) Arts of Vanuatu. University of Hawaii Press.; Davenport WH (1969) Social organization notes on the Northern Santa Cruz Islands: the Main Reef Islands. Baessler-Archiv, Neue Folge 17(1):151-243.; Facey EE (1981) Hereditary chiefship in Nguna. Vanuatu: Politics, Economics and Ritual in Island Melanesia, ed Allen M (Academic Press, Sydney, Australia), pp 295-314. Facey EE (1991) Nguna. Encyclopaedia of World Cultures (Vol II: Oceania) (G.K. Hall and Co, New York, NY), pp 242-244.; Humphreys CB (1926) Southern New Hebrides: An Ethnological Record (Cambridge Univ Press, Cambridge, UK). Spriggs M, Wickler S (1989) Archaeological Research on Erromango: Recent Data on Southern Melanesian Prehistory. Bulletin of the Indo-Pacific Prehistory Association 9:68-91.; Capell A (1958) Culture and Language of Futuna and Aniwa, New Hebrides (Univ Sydney, Sydney, Australia).; Bonnemaison, J (1996) The Art of Power. In Bonnemaison (eds) Arts of Vanuatu. University of Hawaii Press.; Bonnemaison, J (1996) The Art of Power. In Bonnemaison (eds) Arts of Vanuatu. University of Hawaii Press.; Bonnemaison, J (1996) The Art of Power. In Bonnemaison (eds) Arts of Vanuatu. University of Hawaii Press.; Bonnemaison, J (1996) The Art of Power. In Bonnemaison (eds) Arts of Vanuatu. University of Hawaii Press.; Deacon, A. B. 1934. Malekula.; Bonnemaison, J (1996) The Art of Power. In Bonnemaison (eds) Arts of Vanuatu. University of Hawaii Press.; Bonnemaison, J (1996) The Art of Power. In Bonnemaison (eds) Arts of Vanuatu. University of Hawaii Press.; Bonnemaison, J (1996) The Art of Power. In Bonnemaison (eds) Arts of Vanuatu. University of Hawaii Press.; Lane, R. B. 1956. The Heathen Communities of Southeast Pentecost. Journal de la Soci‚te des Oceanistes 12. 139-180., Lane, R. B. 1965. The Melanesians of South Pentecost. In P. Lawrence and M. G. Meggitt (eds.), Gods, Ghosts and Men in Melanesia, 250-279. Lane, R. B., and B. S. Lane. 1957. Unpublished field notes.; Bonnemaison, J (1996) The Art of Power. In Bonnemaison (eds) Arts of Vanuatu. University of Hawaii Press.; Lindström, Lamont (1991) Ajie. Encyclopaedia of World Cultures (Vol II: Oceania) (G.K. Hall and Co, New York, NY), pp 314.; Kirch PV (1994) The Wet and the Dry: Irrigation and Agricultural Intensification in Polynesia (Univ Chicago Press, Chicago, IL). Sahlins MD (1958) Social Stratification in Polynesia (Univ Washington Press, Seattle, WA). Firth R (1939) Primitive Polynesian Economy (George Routledge and Sons, London, UK). Firth R (1959) Social Change in Tikopia: Re-Study of a Polynesian Community after a Generation (Allen and Unwin, London, UK). Firth R (1991) Tikopia. Encyclopaedia of World Cultures (Vol II: Oceania) (G.K. Hall and Co, New York, NY), pp 324-327.; Bonnemaison, J (1996) The Art of Power. In Bonnemaison (eds) Arts of Vanuatu. University of Hawaii Press.; Bonnemaison, J (1996) The Art of Power. In Bonnemaison (eds) Arts of Vanuatu. University of Hawaii Press. \\ 
  Vanuatu and Temotu & Pentecost & 1.000 & saaa1241 & Bonnemaison, J. (1972). Système de grades et diff$\backslash$'\{e\}rences r$\backslash$'\{e\}gionales en Aoba (Nouvelles H$\backslash$'\{e\}brides). Cahiers ORSTOM. S$\backslash$'\{e\}rie Sciences Humaines, 9(1), 87-108.; Tonkinson R (1981) Church and Kastom in Southeast Ambrym. Vanuatu: Politics, Economics and Ritual in Island Melanesia, ed Allen M (Academic Press, Sydney, Australia), pp 237-267.; Humphreys CB (1926) Southern New Hebrides: An Ethnological Record (Cambridge Univ Press, Cambridge, UK); Spriggs M (1982) Taro Cropping Systems in the Southeast Asian-Pacific Region: Archaeological Evidence. Archaeol Ocean 17(1):7-15; Spriggs M (1986) Landscape, Land Use, and Political Transformation in Southern Melanesia. Island Societies: Archaeological Approaches to Evolution and Transformation, ed Kirch PV (Cambridge Univ Press, New York, NY), pp 6-19.; Feinberg R (1988) Socio-Spatial Symbolism and the Logic of Rank on Two Polynesian Outliers. Ethnology 27(3):291-310; Feinberg R (1991) Anuta. Encyclopaedia of World Cultures (Vol II: Oceania) (G.K. Hall and Co, New York, NY), pp 13-16.; Kirch PV (2002) Te Kai Paka-Anuta: Food in a Polynesian Outlier Society. Le Journal de la Soci$\backslash$'\{e\}t$\backslash$'\{e\} des Oc$\backslash$'\{e\}anistes 114-115:71-89.; Bonnemaison, J (1996) The Art of Power. In Bonnemaison (eds) Arts of Vanuatu. University of Hawaii Press.; Davenport WH (1969) Social organization notes on the Northern Santa Cruz Islands: the Main Reef Islands. Baessler-Archiv, Neue Folge 17(1):151-243.; Facey EE (1981) Hereditary chiefship in Nguna. Vanuatu: Politics, Economics and Ritual in Island Melanesia, ed Allen M (Academic Press, Sydney, Australia), pp 295-314. Facey EE (1991) Nguna. Encyclopaedia of World Cultures (Vol II: Oceania) (G.K. Hall and Co, New York, NY), pp 242-244.; Humphreys CB (1926) Southern New Hebrides: An Ethnological Record (Cambridge Univ Press, Cambridge, UK). Spriggs M, Wickler S (1989) Archaeological Research on Erromango: Recent Data on Southern Melanesian Prehistory. Bulletin of the Indo-Pacific Prehistory Association 9:68-91.; Capell A (1958) Culture and Language of Futuna and Aniwa, New Hebrides (Univ Sydney, Sydney, Australia).; Bonnemaison, J (1996) The Art of Power. In Bonnemaison (eds) Arts of Vanuatu. University of Hawaii Press.; Bonnemaison, J (1996) The Art of Power. In Bonnemaison (eds) Arts of Vanuatu. University of Hawaii Press.; Bonnemaison, J (1996) The Art of Power. In Bonnemaison (eds) Arts of Vanuatu. University of Hawaii Press.; Bonnemaison, J (1996) The Art of Power. In Bonnemaison (eds) Arts of Vanuatu. University of Hawaii Press.; Deacon, A. B. 1934. Malekula.; Bonnemaison, J (1996) The Art of Power. In Bonnemaison (eds) Arts of Vanuatu. University of Hawaii Press.; Bonnemaison, J (1996) The Art of Power. In Bonnemaison (eds) Arts of Vanuatu. University of Hawaii Press.; Bonnemaison, J (1996) The Art of Power. In Bonnemaison (eds) Arts of Vanuatu. University of Hawaii Press.; Lane, R. B. 1956. The Heathen Communities of Southeast Pentecost. Journal de la Soci‚te des Oceanistes 12. 139-180., Lane, R. B. 1965. The Melanesians of South Pentecost. In P. Lawrence and M. G. Meggitt (eds.), Gods, Ghosts and Men in Melanesia, 250-279. Lane, R. B., and B. S. Lane. 1957. Unpublished field notes.; Bonnemaison, J (1996) The Art of Power. In Bonnemaison (eds) Arts of Vanuatu. University of Hawaii Press.; Lindström, Lamont (1991) Ajie. Encyclopaedia of World Cultures (Vol II: Oceania) (G.K. Hall and Co, New York, NY), pp 314.; Kirch PV (1994) The Wet and the Dry: Irrigation and Agricultural Intensification in Polynesia (Univ Chicago Press, Chicago, IL). Sahlins MD (1958) Social Stratification in Polynesia (Univ Washington Press, Seattle, WA). Firth R (1939) Primitive Polynesian Economy (George Routledge and Sons, London, UK). Firth R (1959) Social Change in Tikopia: Re-Study of a Polynesian Community after a Generation (Allen and Unwin, London, UK). Firth R (1991) Tikopia. Encyclopaedia of World Cultures (Vol II: Oceania) (G.K. Hall and Co, New York, NY), pp 324-327.; Bonnemaison, J (1996) The Art of Power. In Bonnemaison (eds) Arts of Vanuatu. University of Hawaii Press.; Bonnemaison, J (1996) The Art of Power. In Bonnemaison (eds) Arts of Vanuatu. University of Hawaii Press. \\ 
  Vanuatu and Temotu & Santo & 1.000 & butm1237, polo1242, akei1237, ambl1237, fort1240, lore1244, mere1242, moro1286, noku1237, piam1242, rori1237, saka1289, shar1244, tamb1253, tasm1246, tial1239, tolo1255, valp1237, wail1242, wusi1237, tang1347, tang1347 & Bonnemaison, J. (1972). Système de grades et diff$\backslash$'\{e\}rences r$\backslash$'\{e\}gionales en Aoba (Nouvelles H$\backslash$'\{e\}brides). Cahiers ORSTOM. S$\backslash$'\{e\}rie Sciences Humaines, 9(1), 87-108.; Tonkinson R (1981) Church and Kastom in Southeast Ambrym. Vanuatu: Politics, Economics and Ritual in Island Melanesia, ed Allen M (Academic Press, Sydney, Australia), pp 237-267.; Humphreys CB (1926) Southern New Hebrides: An Ethnological Record (Cambridge Univ Press, Cambridge, UK); Spriggs M (1982) Taro Cropping Systems in the Southeast Asian-Pacific Region: Archaeological Evidence. Archaeol Ocean 17(1):7-15; Spriggs M (1986) Landscape, Land Use, and Political Transformation in Southern Melanesia. Island Societies: Archaeological Approaches to Evolution and Transformation, ed Kirch PV (Cambridge Univ Press, New York, NY), pp 6-19.; Feinberg R (1988) Socio-Spatial Symbolism and the Logic of Rank on Two Polynesian Outliers. Ethnology 27(3):291-310; Feinberg R (1991) Anuta. Encyclopaedia of World Cultures (Vol II: Oceania) (G.K. Hall and Co, New York, NY), pp 13-16.; Kirch PV (2002) Te Kai Paka-Anuta: Food in a Polynesian Outlier Society. Le Journal de la Soci$\backslash$'\{e\}t$\backslash$'\{e\} des Oc$\backslash$'\{e\}anistes 114-115:71-89.; Bonnemaison, J (1996) The Art of Power. In Bonnemaison (eds) Arts of Vanuatu. University of Hawaii Press.; Davenport WH (1969) Social organization notes on the Northern Santa Cruz Islands: the Main Reef Islands. Baessler-Archiv, Neue Folge 17(1):151-243.; Facey EE (1981) Hereditary chiefship in Nguna. Vanuatu: Politics, Economics and Ritual in Island Melanesia, ed Allen M (Academic Press, Sydney, Australia), pp 295-314. Facey EE (1991) Nguna. Encyclopaedia of World Cultures (Vol II: Oceania) (G.K. Hall and Co, New York, NY), pp 242-244.; Humphreys CB (1926) Southern New Hebrides: An Ethnological Record (Cambridge Univ Press, Cambridge, UK). Spriggs M, Wickler S (1989) Archaeological Research on Erromango: Recent Data on Southern Melanesian Prehistory. Bulletin of the Indo-Pacific Prehistory Association 9:68-91.; Capell A (1958) Culture and Language of Futuna and Aniwa, New Hebrides (Univ Sydney, Sydney, Australia).; Bonnemaison, J (1996) The Art of Power. In Bonnemaison (eds) Arts of Vanuatu. University of Hawaii Press.; Bonnemaison, J (1996) The Art of Power. In Bonnemaison (eds) Arts of Vanuatu. University of Hawaii Press.; Bonnemaison, J (1996) The Art of Power. In Bonnemaison (eds) Arts of Vanuatu. University of Hawaii Press.; Bonnemaison, J (1996) The Art of Power. In Bonnemaison (eds) Arts of Vanuatu. University of Hawaii Press.; Deacon, A. B. 1934. Malekula.; Bonnemaison, J (1996) The Art of Power. In Bonnemaison (eds) Arts of Vanuatu. University of Hawaii Press.; Bonnemaison, J (1996) The Art of Power. In Bonnemaison (eds) Arts of Vanuatu. University of Hawaii Press.; Bonnemaison, J (1996) The Art of Power. In Bonnemaison (eds) Arts of Vanuatu. University of Hawaii Press.; Lane, R. B. 1956. The Heathen Communities of Southeast Pentecost. Journal de la Soci‚te des Oceanistes 12. 139-180., Lane, R. B. 1965. The Melanesians of South Pentecost. In P. Lawrence and M. G. Meggitt (eds.), Gods, Ghosts and Men in Melanesia, 250-279. Lane, R. B., and B. S. Lane. 1957. Unpublished field notes.; Bonnemaison, J (1996) The Art of Power. In Bonnemaison (eds) Arts of Vanuatu. University of Hawaii Press.; Lindström, Lamont (1991) Ajie. Encyclopaedia of World Cultures (Vol II: Oceania) (G.K. Hall and Co, New York, NY), pp 314.; Kirch PV (1994) The Wet and the Dry: Irrigation and Agricultural Intensification in Polynesia (Univ Chicago Press, Chicago, IL). Sahlins MD (1958) Social Stratification in Polynesia (Univ Washington Press, Seattle, WA). Firth R (1939) Primitive Polynesian Economy (George Routledge and Sons, London, UK). Firth R (1959) Social Change in Tikopia: Re-Study of a Polynesian Community after a Generation (Allen and Unwin, London, UK). Firth R (1991) Tikopia. Encyclopaedia of World Cultures (Vol II: Oceania) (G.K. Hall and Co, New York, NY), pp 324-327.; Bonnemaison, J (1996) The Art of Power. In Bonnemaison (eds) Arts of Vanuatu. University of Hawaii Press.; Bonnemaison, J (1996) The Art of Power. In Bonnemaison (eds) Arts of Vanuatu. University of Hawaii Press. \\ 
  Vanuatu and Temotu & Tanna & 1.000 & sout2869 & Bonnemaison, J. (1972). Système de grades et diff$\backslash$'\{e\}rences r$\backslash$'\{e\}gionales en Aoba (Nouvelles H$\backslash$'\{e\}brides). Cahiers ORSTOM. S$\backslash$'\{e\}rie Sciences Humaines, 9(1), 87-108.; Tonkinson R (1981) Church and Kastom in Southeast Ambrym. Vanuatu: Politics, Economics and Ritual in Island Melanesia, ed Allen M (Academic Press, Sydney, Australia), pp 237-267.; Humphreys CB (1926) Southern New Hebrides: An Ethnological Record (Cambridge Univ Press, Cambridge, UK); Spriggs M (1982) Taro Cropping Systems in the Southeast Asian-Pacific Region: Archaeological Evidence. Archaeol Ocean 17(1):7-15; Spriggs M (1986) Landscape, Land Use, and Political Transformation in Southern Melanesia. Island Societies: Archaeological Approaches to Evolution and Transformation, ed Kirch PV (Cambridge Univ Press, New York, NY), pp 6-19.; Feinberg R (1988) Socio-Spatial Symbolism and the Logic of Rank on Two Polynesian Outliers. Ethnology 27(3):291-310; Feinberg R (1991) Anuta. Encyclopaedia of World Cultures (Vol II: Oceania) (G.K. Hall and Co, New York, NY), pp 13-16.; Kirch PV (2002) Te Kai Paka-Anuta: Food in a Polynesian Outlier Society. Le Journal de la Soci$\backslash$'\{e\}t$\backslash$'\{e\} des Oc$\backslash$'\{e\}anistes 114-115:71-89.; Bonnemaison, J (1996) The Art of Power. In Bonnemaison (eds) Arts of Vanuatu. University of Hawaii Press.; Davenport WH (1969) Social organization notes on the Northern Santa Cruz Islands: the Main Reef Islands. Baessler-Archiv, Neue Folge 17(1):151-243.; Facey EE (1981) Hereditary chiefship in Nguna. Vanuatu: Politics, Economics and Ritual in Island Melanesia, ed Allen M (Academic Press, Sydney, Australia), pp 295-314. Facey EE (1991) Nguna. Encyclopaedia of World Cultures (Vol II: Oceania) (G.K. Hall and Co, New York, NY), pp 242-244.; Humphreys CB (1926) Southern New Hebrides: An Ethnological Record (Cambridge Univ Press, Cambridge, UK). Spriggs M, Wickler S (1989) Archaeological Research on Erromango: Recent Data on Southern Melanesian Prehistory. Bulletin of the Indo-Pacific Prehistory Association 9:68-91.; Capell A (1958) Culture and Language of Futuna and Aniwa, New Hebrides (Univ Sydney, Sydney, Australia).; Bonnemaison, J (1996) The Art of Power. In Bonnemaison (eds) Arts of Vanuatu. University of Hawaii Press.; Bonnemaison, J (1996) The Art of Power. In Bonnemaison (eds) Arts of Vanuatu. University of Hawaii Press.; Bonnemaison, J (1996) The Art of Power. In Bonnemaison (eds) Arts of Vanuatu. University of Hawaii Press.; Bonnemaison, J (1996) The Art of Power. In Bonnemaison (eds) Arts of Vanuatu. University of Hawaii Press.; Deacon, A. B. 1934. Malekula.; Bonnemaison, J (1996) The Art of Power. In Bonnemaison (eds) Arts of Vanuatu. University of Hawaii Press.; Bonnemaison, J (1996) The Art of Power. In Bonnemaison (eds) Arts of Vanuatu. University of Hawaii Press.; Bonnemaison, J (1996) The Art of Power. In Bonnemaison (eds) Arts of Vanuatu. University of Hawaii Press.; Lane, R. B. 1956. The Heathen Communities of Southeast Pentecost. Journal de la Soci‚te des Oceanistes 12. 139-180., Lane, R. B. 1965. The Melanesians of South Pentecost. In P. Lawrence and M. G. Meggitt (eds.), Gods, Ghosts and Men in Melanesia, 250-279. Lane, R. B., and B. S. Lane. 1957. Unpublished field notes.; Bonnemaison, J (1996) The Art of Power. In Bonnemaison (eds) Arts of Vanuatu. University of Hawaii Press.; Lindström, Lamont (1991) Ajie. Encyclopaedia of World Cultures (Vol II: Oceania) (G.K. Hall and Co, New York, NY), pp 314.; Kirch PV (1994) The Wet and the Dry: Irrigation and Agricultural Intensification in Polynesia (Univ Chicago Press, Chicago, IL). Sahlins MD (1958) Social Stratification in Polynesia (Univ Washington Press, Seattle, WA). Firth R (1939) Primitive Polynesian Economy (George Routledge and Sons, London, UK). Firth R (1959) Social Change in Tikopia: Re-Study of a Polynesian Community after a Generation (Allen and Unwin, London, UK). Firth R (1991) Tikopia. Encyclopaedia of World Cultures (Vol II: Oceania) (G.K. Hall and Co, New York, NY), pp 324-327.; Bonnemaison, J (1996) The Art of Power. In Bonnemaison (eds) Arts of Vanuatu. University of Hawaii Press.; Bonnemaison, J (1996) The Art of Power. In Bonnemaison (eds) Arts of Vanuatu. University of Hawaii Press. \\ 
  Vanuatu and Temotu & Tikopia & 2.000 & tiko1237 & Bonnemaison, J. (1972). Système de grades et diff$\backslash$'\{e\}rences r$\backslash$'\{e\}gionales en Aoba (Nouvelles H$\backslash$'\{e\}brides). Cahiers ORSTOM. S$\backslash$'\{e\}rie Sciences Humaines, 9(1), 87-108.; Tonkinson R (1981) Church and Kastom in Southeast Ambrym. Vanuatu: Politics, Economics and Ritual in Island Melanesia, ed Allen M (Academic Press, Sydney, Australia), pp 237-267.; Humphreys CB (1926) Southern New Hebrides: An Ethnological Record (Cambridge Univ Press, Cambridge, UK); Spriggs M (1982) Taro Cropping Systems in the Southeast Asian-Pacific Region: Archaeological Evidence. Archaeol Ocean 17(1):7-15; Spriggs M (1986) Landscape, Land Use, and Political Transformation in Southern Melanesia. Island Societies: Archaeological Approaches to Evolution and Transformation, ed Kirch PV (Cambridge Univ Press, New York, NY), pp 6-19.; Feinberg R (1988) Socio-Spatial Symbolism and the Logic of Rank on Two Polynesian Outliers. Ethnology 27(3):291-310; Feinberg R (1991) Anuta. Encyclopaedia of World Cultures (Vol II: Oceania) (G.K. Hall and Co, New York, NY), pp 13-16.; Kirch PV (2002) Te Kai Paka-Anuta: Food in a Polynesian Outlier Society. Le Journal de la Soci$\backslash$'\{e\}t$\backslash$'\{e\} des Oc$\backslash$'\{e\}anistes 114-115:71-89.; Bonnemaison, J (1996) The Art of Power. In Bonnemaison (eds) Arts of Vanuatu. University of Hawaii Press.; Davenport WH (1969) Social organization notes on the Northern Santa Cruz Islands: the Main Reef Islands. Baessler-Archiv, Neue Folge 17(1):151-243.; Facey EE (1981) Hereditary chiefship in Nguna. Vanuatu: Politics, Economics and Ritual in Island Melanesia, ed Allen M (Academic Press, Sydney, Australia), pp 295-314. Facey EE (1991) Nguna. Encyclopaedia of World Cultures (Vol II: Oceania) (G.K. Hall and Co, New York, NY), pp 242-244.; Humphreys CB (1926) Southern New Hebrides: An Ethnological Record (Cambridge Univ Press, Cambridge, UK). Spriggs M, Wickler S (1989) Archaeological Research on Erromango: Recent Data on Southern Melanesian Prehistory. Bulletin of the Indo-Pacific Prehistory Association 9:68-91.; Capell A (1958) Culture and Language of Futuna and Aniwa, New Hebrides (Univ Sydney, Sydney, Australia).; Bonnemaison, J (1996) The Art of Power. In Bonnemaison (eds) Arts of Vanuatu. University of Hawaii Press.; Bonnemaison, J (1996) The Art of Power. In Bonnemaison (eds) Arts of Vanuatu. University of Hawaii Press.; Bonnemaison, J (1996) The Art of Power. In Bonnemaison (eds) Arts of Vanuatu. University of Hawaii Press.; Bonnemaison, J (1996) The Art of Power. In Bonnemaison (eds) Arts of Vanuatu. University of Hawaii Press.; Deacon, A. B. 1934. Malekula.; Bonnemaison, J (1996) The Art of Power. In Bonnemaison (eds) Arts of Vanuatu. University of Hawaii Press.; Bonnemaison, J (1996) The Art of Power. In Bonnemaison (eds) Arts of Vanuatu. University of Hawaii Press.; Bonnemaison, J (1996) The Art of Power. In Bonnemaison (eds) Arts of Vanuatu. University of Hawaii Press.; Lane, R. B. 1956. The Heathen Communities of Southeast Pentecost. Journal de la Soci‚te des Oceanistes 12. 139-180., Lane, R. B. 1965. The Melanesians of South Pentecost. In P. Lawrence and M. G. Meggitt (eds.), Gods, Ghosts and Men in Melanesia, 250-279. Lane, R. B., and B. S. Lane. 1957. Unpublished field notes.; Bonnemaison, J (1996) The Art of Power. In Bonnemaison (eds) Arts of Vanuatu. University of Hawaii Press.; Lindström, Lamont (1991) Ajie. Encyclopaedia of World Cultures (Vol II: Oceania) (G.K. Hall and Co, New York, NY), pp 314.; Kirch PV (1994) The Wet and the Dry: Irrigation and Agricultural Intensification in Polynesia (Univ Chicago Press, Chicago, IL). Sahlins MD (1958) Social Stratification in Polynesia (Univ Washington Press, Seattle, WA). Firth R (1939) Primitive Polynesian Economy (George Routledge and Sons, London, UK). Firth R (1959) Social Change in Tikopia: Re-Study of a Polynesian Community after a Generation (Allen and Unwin, London, UK). Firth R (1991) Tikopia. Encyclopaedia of World Cultures (Vol II: Oceania) (G.K. Hall and Co, New York, NY), pp 324-327.; Bonnemaison, J (1996) The Art of Power. In Bonnemaison (eds) Arts of Vanuatu. University of Hawaii Press.; Bonnemaison, J (1996) The Art of Power. In Bonnemaison (eds) Arts of Vanuatu. University of Hawaii Press. \\ 
  Vanuatu and Temotu & Ureparapara & 1.000 & leha1243, leha1244 & Bonnemaison, J. (1972). Système de grades et diff$\backslash$'\{e\}rences r$\backslash$'\{e\}gionales en Aoba (Nouvelles H$\backslash$'\{e\}brides). Cahiers ORSTOM. S$\backslash$'\{e\}rie Sciences Humaines, 9(1), 87-108.; Tonkinson R (1981) Church and Kastom in Southeast Ambrym. Vanuatu: Politics, Economics and Ritual in Island Melanesia, ed Allen M (Academic Press, Sydney, Australia), pp 237-267.; Humphreys CB (1926) Southern New Hebrides: An Ethnological Record (Cambridge Univ Press, Cambridge, UK); Spriggs M (1982) Taro Cropping Systems in the Southeast Asian-Pacific Region: Archaeological Evidence. Archaeol Ocean 17(1):7-15; Spriggs M (1986) Landscape, Land Use, and Political Transformation in Southern Melanesia. Island Societies: Archaeological Approaches to Evolution and Transformation, ed Kirch PV (Cambridge Univ Press, New York, NY), pp 6-19.; Feinberg R (1988) Socio-Spatial Symbolism and the Logic of Rank on Two Polynesian Outliers. Ethnology 27(3):291-310; Feinberg R (1991) Anuta. Encyclopaedia of World Cultures (Vol II: Oceania) (G.K. Hall and Co, New York, NY), pp 13-16.; Kirch PV (2002) Te Kai Paka-Anuta: Food in a Polynesian Outlier Society. Le Journal de la Soci$\backslash$'\{e\}t$\backslash$'\{e\} des Oc$\backslash$'\{e\}anistes 114-115:71-89.; Bonnemaison, J (1996) The Art of Power. In Bonnemaison (eds) Arts of Vanuatu. University of Hawaii Press.; Davenport WH (1969) Social organization notes on the Northern Santa Cruz Islands: the Main Reef Islands. Baessler-Archiv, Neue Folge 17(1):151-243.; Facey EE (1981) Hereditary chiefship in Nguna. Vanuatu: Politics, Economics and Ritual in Island Melanesia, ed Allen M (Academic Press, Sydney, Australia), pp 295-314. Facey EE (1991) Nguna. Encyclopaedia of World Cultures (Vol II: Oceania) (G.K. Hall and Co, New York, NY), pp 242-244.; Humphreys CB (1926) Southern New Hebrides: An Ethnological Record (Cambridge Univ Press, Cambridge, UK). Spriggs M, Wickler S (1989) Archaeological Research on Erromango: Recent Data on Southern Melanesian Prehistory. Bulletin of the Indo-Pacific Prehistory Association 9:68-91.; Capell A (1958) Culture and Language of Futuna and Aniwa, New Hebrides (Univ Sydney, Sydney, Australia).; Bonnemaison, J (1996) The Art of Power. In Bonnemaison (eds) Arts of Vanuatu. University of Hawaii Press.; Bonnemaison, J (1996) The Art of Power. In Bonnemaison (eds) Arts of Vanuatu. University of Hawaii Press.; Bonnemaison, J (1996) The Art of Power. In Bonnemaison (eds) Arts of Vanuatu. University of Hawaii Press.; Bonnemaison, J (1996) The Art of Power. In Bonnemaison (eds) Arts of Vanuatu. University of Hawaii Press.; Deacon, A. B. 1934. Malekula.; Bonnemaison, J (1996) The Art of Power. In Bonnemaison (eds) Arts of Vanuatu. University of Hawaii Press.; Bonnemaison, J (1996) The Art of Power. In Bonnemaison (eds) Arts of Vanuatu. University of Hawaii Press.; Bonnemaison, J (1996) The Art of Power. In Bonnemaison (eds) Arts of Vanuatu. University of Hawaii Press.; Lane, R. B. 1956. The Heathen Communities of Southeast Pentecost. Journal de la Soci‚te des Oceanistes 12. 139-180., Lane, R. B. 1965. The Melanesians of South Pentecost. In P. Lawrence and M. G. Meggitt (eds.), Gods, Ghosts and Men in Melanesia, 250-279. Lane, R. B., and B. S. Lane. 1957. Unpublished field notes.; Bonnemaison, J (1996) The Art of Power. In Bonnemaison (eds) Arts of Vanuatu. University of Hawaii Press.; Lindström, Lamont (1991) Ajie. Encyclopaedia of World Cultures (Vol II: Oceania) (G.K. Hall and Co, New York, NY), pp 314.; Kirch PV (1994) The Wet and the Dry: Irrigation and Agricultural Intensification in Polynesia (Univ Chicago Press, Chicago, IL). Sahlins MD (1958) Social Stratification in Polynesia (Univ Washington Press, Seattle, WA). Firth R (1939) Primitive Polynesian Economy (George Routledge and Sons, London, UK). Firth R (1959) Social Change in Tikopia: Re-Study of a Polynesian Community after a Generation (Allen and Unwin, London, UK). Firth R (1991) Tikopia. Encyclopaedia of World Cultures (Vol II: Oceania) (G.K. Hall and Co, New York, NY), pp 324-327.; Bonnemaison, J (1996) The Art of Power. In Bonnemaison (eds) Arts of Vanuatu. University of Hawaii Press.; Bonnemaison, J (1996) The Art of Power. In Bonnemaison (eds) Arts of Vanuatu. University of Hawaii Press. \\ 
  Vanuatu and Temotu & Vanua Lava & 1.000 & leme1238, vera1241, vure1239 & Bonnemaison, J. (1972). Système de grades et diff$\backslash$'\{e\}rences r$\backslash$'\{e\}gionales en Aoba (Nouvelles H$\backslash$'\{e\}brides). Cahiers ORSTOM. S$\backslash$'\{e\}rie Sciences Humaines, 9(1), 87-108.; Tonkinson R (1981) Church and Kastom in Southeast Ambrym. Vanuatu: Politics, Economics and Ritual in Island Melanesia, ed Allen M (Academic Press, Sydney, Australia), pp 237-267.; Humphreys CB (1926) Southern New Hebrides: An Ethnological Record (Cambridge Univ Press, Cambridge, UK); Spriggs M (1982) Taro Cropping Systems in the Southeast Asian-Pacific Region: Archaeological Evidence. Archaeol Ocean 17(1):7-15; Spriggs M (1986) Landscape, Land Use, and Political Transformation in Southern Melanesia. Island Societies: Archaeological Approaches to Evolution and Transformation, ed Kirch PV (Cambridge Univ Press, New York, NY), pp 6-19.; Feinberg R (1988) Socio-Spatial Symbolism and the Logic of Rank on Two Polynesian Outliers. Ethnology 27(3):291-310; Feinberg R (1991) Anuta. Encyclopaedia of World Cultures (Vol II: Oceania) (G.K. Hall and Co, New York, NY), pp 13-16.; Kirch PV (2002) Te Kai Paka-Anuta: Food in a Polynesian Outlier Society. Le Journal de la Soci$\backslash$'\{e\}t$\backslash$'\{e\} des Oc$\backslash$'\{e\}anistes 114-115:71-89.; Bonnemaison, J (1996) The Art of Power. In Bonnemaison (eds) Arts of Vanuatu. University of Hawaii Press.; Davenport WH (1969) Social organization notes on the Northern Santa Cruz Islands: the Main Reef Islands. Baessler-Archiv, Neue Folge 17(1):151-243.; Facey EE (1981) Hereditary chiefship in Nguna. Vanuatu: Politics, Economics and Ritual in Island Melanesia, ed Allen M (Academic Press, Sydney, Australia), pp 295-314. Facey EE (1991) Nguna. Encyclopaedia of World Cultures (Vol II: Oceania) (G.K. Hall and Co, New York, NY), pp 242-244.; Humphreys CB (1926) Southern New Hebrides: An Ethnological Record (Cambridge Univ Press, Cambridge, UK). Spriggs M, Wickler S (1989) Archaeological Research on Erromango: Recent Data on Southern Melanesian Prehistory. Bulletin of the Indo-Pacific Prehistory Association 9:68-91.; Capell A (1958) Culture and Language of Futuna and Aniwa, New Hebrides (Univ Sydney, Sydney, Australia).; Bonnemaison, J (1996) The Art of Power. In Bonnemaison (eds) Arts of Vanuatu. University of Hawaii Press.; Bonnemaison, J (1996) The Art of Power. In Bonnemaison (eds) Arts of Vanuatu. University of Hawaii Press.; Bonnemaison, J (1996) The Art of Power. In Bonnemaison (eds) Arts of Vanuatu. University of Hawaii Press.; Bonnemaison, J (1996) The Art of Power. In Bonnemaison (eds) Arts of Vanuatu. University of Hawaii Press.; Deacon, A. B. 1934. Malekula.; Bonnemaison, J (1996) The Art of Power. In Bonnemaison (eds) Arts of Vanuatu. University of Hawaii Press.; Bonnemaison, J (1996) The Art of Power. In Bonnemaison (eds) Arts of Vanuatu. University of Hawaii Press.; Bonnemaison, J (1996) The Art of Power. In Bonnemaison (eds) Arts of Vanuatu. University of Hawaii Press.; Lane, R. B. 1956. The Heathen Communities of Southeast Pentecost. Journal de la Soci‚te des Oceanistes 12. 139-180., Lane, R. B. 1965. The Melanesians of South Pentecost. In P. Lawrence and M. G. Meggitt (eds.), Gods, Ghosts and Men in Melanesia, 250-279. Lane, R. B., and B. S. Lane. 1957. Unpublished field notes.; Bonnemaison, J (1996) The Art of Power. In Bonnemaison (eds) Arts of Vanuatu. University of Hawaii Press.; Lindström, Lamont (1991) Ajie. Encyclopaedia of World Cultures (Vol II: Oceania) (G.K. Hall and Co, New York, NY), pp 314.; Kirch PV (1994) The Wet and the Dry: Irrigation and Agricultural Intensification in Polynesia (Univ Chicago Press, Chicago, IL). Sahlins MD (1958) Social Stratification in Polynesia (Univ Washington Press, Seattle, WA). Firth R (1939) Primitive Polynesian Economy (George Routledge and Sons, London, UK). Firth R (1959) Social Change in Tikopia: Re-Study of a Polynesian Community after a Generation (Allen and Unwin, London, UK). Firth R (1991) Tikopia. Encyclopaedia of World Cultures (Vol II: Oceania) (G.K. Hall and Co, New York, NY), pp 324-327.; Bonnemaison, J (1996) The Art of Power. In Bonnemaison (eds) Arts of Vanuatu. University of Hawaii Press.; Bonnemaison, J (1996) The Art of Power. In Bonnemaison (eds) Arts of Vanuatu. University of Hawaii Press. \\ 
  Woleai & Woleai & 2.000 & wole1240, wole1240 & Alkire WH (1991) Woleai. Encyclopaedia of World Cultures (Vol II: Oceania) (G.K. Hall and Co, New York, NY), pp 383-384. Burrows EG, Spiro ME (1953) An Atoll Culture: Ethnography of Ifaluk in the Central Carolines (Human Relations Area Files, New Haven, CT). \\ 
  Ānewetak & Ratak + Rālik & 3.000 & mars1254 & Carucci LM (1991) Marshall Islands. Encyclopaedia of World Cultures (Vol II: Oceania) (G.K. Hall and Co, New York, NY), pp 191-194. Erdland A (1961) The Marshall Islanders: Life and Customs, Thought and Religion of a South Seas People (R. Neuse, Trans) (Human Relations Area Files, New Haven, CT) (Originally published 1914). Williamson I, Sabath MD (1982) Island Population, Land Area, and Climate: a Case Study of the Marshall Islands. Hum Ecol 10(1):71-84. \\ 
   \bottomrule
\end{tabular}
\caption{Table of political complexity values (EA033).} 
\label{appendix_pol_complex_xtable}
\end{table}
